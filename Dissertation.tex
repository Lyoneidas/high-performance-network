\documentclass[11pt,openright,a4paper]{report}
%%
%% This document template assumes you will use pdflatex.  If you are using
%% latex and dvipdfm to translate to pdf, insert dvipdfm into the options.
%%

\include{DissertationDefs}    %% These are the includes required for the doc 
\usepackage{url}
\usepackage{graphicx} 
\usepackage{booktabs}
\usepackage{float}
\usepackage{amsmath}
\usepackage[]{algorithm2e}
\usepackage{pdfpages}
\title{High Performance Network Switch Solution Using Open Sourced Hardware}
\author{Zhouqiang Tu}
\date{MSc in Internet System and Security\\The University of Bath\\September 2016}


\begin{document}


% Set this to the language you want to use in your code listings (if any)
\lstset{language=Java,breaklines,breakatwhitespace,basicstyle=\small}


\setcounter{page}{0}
\pagenumbering{roman}


\maketitle
\newpage


% Set this to the number of years consultation prohibition, or 0 if no limit
\consultation{0}
\newpage


\declaration{High Performance Network Switch Solution Using Open Sourced Hardware Accepting Dual Feeds}{Zhouqiang Tu}
\newpage


\abstract
The thesis provides an integrated solution for dual-feed data handler implemented by open sourced hardware. The implementation of the solution is a Beowulf cluster with two data handlers and one data receiver, interconnected by RJ-45 Ethernet cables with Message Passing Interface(MPI) supported intercommunication. The main goal is to prove that commodity hardware can achieve similar system efficiency compared to the established commercial hardware. The cost in implementing ultra-low latency network can be reduced significantly by using the open sourced resources assisted with FPGA. The project focuses on constructing an experimental environment consisted of three Raspberry Pi2 Model B with one Spartan FPGA. The software system is running on Raspbian, a customized Debian OS customized for the Raspberry Pi. The main implementation is a parallel application based on MPICH2 among three ARM CPUs, written in Python, providing basic functions of data receiving,  filtering and error check. The experiments performed in the project shows that the solution reduces network latency significantly with reasonable cost in bandwidth and computational power. The system can be implemented in both high-frequency market data infrastructure and high-performance computational clusters as the distributed switch, for it meets the requirements of: low-latency data transmission, reliable end-to-end communication and acceleration in collective communication in MPI.  
\newpage

\tableofcontents
\newpage
\listoffigures
\newpage
\listoftables
\newpage

\setcounter{page}{1}
\pagenumbering{arabic}



\chapter{Thesis Background and Literature Review}
%% Uncomment this to include a separate tex file wih the introduction contents
\section{Thesis Inspiration}
In the 1970' when the New York Stock Exchange started to implement the computerization on trading floors and accepts electronic orders, financial world had begun the era of high frequency trading(HFT) over market data since then\cite{mcgowan2010rise}. The rapid evolution in computer technologies especially the establishment of world-wide network, turns the traditional speculation trading period from minutes to hours into microsecond level. Financial instruments and organizations in the modern world needs to provide an live update on their latest and updated current market status, and provide a package of information to its members and partners. The data they need to update is includes bid/ask prise, completed trades and other market status\cite{alexander2001market}. This package of information is aggregated, and streamed as 'Market Data Feed'. Modern financial market requires high performance infrastructure due to the implementation of automated trading system, which implements complicate market strategies and uses the data feed as sources\cite{le2009automated}. Modern strategies requires huge computational power to perform mathematical modelling, and high performance network to make the order ahead of any competitors.\\
Low latency infrastructure is invented to meet the market requirements, it is usually consisted of two parts,  computational nodes and network infrastructure. Existing solutions for the former part are customized servers, cloud computer farms and special data centres. They are programmed to perform data analysis over live market data feed, the outputs are trading behaviours packed as orders . Firms in computer network services are focusing on the latter part, as the latency in this part can be solved by engineering practises.\\
Solutions for low latency network usually come with high cost non-scalable hardware, and requires racks of server cabinets to install the devices. However, the entry barrier in the upfront investment for network infrastructure makes it difficult for medium or small financial institutions to start business in this field.\\
Researchers are looking forward to find alternative solutions to build network infrastructures that supports low latency market data feed. The researchers in high performance computation cluster provides a possible solution. The interconnect networks in supercomputers follows a different design in network interfaces. The improving design of parallel program architectural in supercomputers requires more efficient parallelism in network interface\cite{pang2014th}. We introduces the high scalable design and the fat-tree topology in the architecture into our design of high performance network switches, and achieve similar acceleration in data process as the existing commercial solutions.\\

\section{Literature review}
\subsection{Market Data Feed}
Market data feed is the network data provided by the market data providers, and consumed by participators in high-frequency trading field. One importance feature is the microsecond-level updating frequency of the data. We focus on the network infrastructure behind it to implement the high-frequency data streaming, and long range transportation as modern traders are based globally and making trade deals 24 hours$\times$7 days\cite{chlistalla2011high}.\\
\begin{figure}[H] 
	\centering\includegraphics[width=0.7\linewidth]{picture/tradevenue.PNG} 
	\caption{Relationship among trade venue, data vendors and customers}
	\label{fig:1} 
\end{figure} 
The low latency market data usually carries users level data including current price information and trader status, the individual portfolio valuation, financial market data, trading activities and alarms. A matrix of these data forms an 'Order Book'\cite{menkveld2013high}. The data feeds have different price for varies delays, famous market data feed providers, or the data vendors, such as Yahoo finance, ThomsonReuters and Bloomberg provide combinations of pricing and other market information in different latencies\cite{hasbrouck2013low}. For example, data delayed over 5 minutes can be fetched from Yahoo finance Application Programming Interface(API) for free\cite{financeyahoo}, and ThomsonReuters provide paid services in different level of delay. Customers can choose to use designed data feed services to meet their business requirements, show in the reference to picture \ref{fig:1}.\\
The Market Data feeds usually transmitted in User Datagram Protocol(UDP) datagrams\cite{brook2015low}.in order to achieve high frequency data transmission, the size of each datagram is not big, and usually will not exceed the limit of one UDP datagram.\\
Most of the high frequency trading companies uses more than one data feeds to ensure the data quality, and technology companies have implemented some technologies different to the normal global data networks.\\
\subsection{Low Latency Network Infrastructure}
Engineers provide solutions to achieve lower latency in network transmission for market data, including direct redundant fibre cables from trading centre to traders, shortest path transmission and dual data feed. For example, solution using microwave towers reducing the cost of digging tunnels for fibres and enabling signals transmit over the sphere is implemented between New Jersey and Chicago\cite{htfbackyard}, shows in the picture \ref{fig:2}. Engineers also developed 'dual-feed' mechanism to reduce the cost in retransmission when package lost\cite{zusman1999fault}.The introduction of dual feed, or the A-B feed of market data transmission uses two feeds with offset in delay to avoid significant failure such as delay or fault, especially for regional traders who want to trade across long geographical distances. \\
The low latency network is consisted of at least two hardware infrastructures: the high speed network equipment to carry the signals usually linked by optic fibres, for example, the 10Gb Ethernet; the Global Position System(GPS) based clock system to synchronize both sides\cite{brook2015low}.\\
Among all the technologies implemented, the A-B data feed is the most widely used mechanism to reduce network latency.\\
\begin{figure}[H]
	\centering\includegraphics[width=0.7\linewidth]{picture/newyork-chicago.jpg}
	\caption{Chicago-New Jersey microwave networks from data vendor Quincy Data}
	\label{fig:2}
\end{figure}
\subsection{User Datagram Protocol}
As discussed above, high-frequency market data uses User Datagram Protocol as the communication protocol forming the low-latency network. It is necessary to analyse the UDP datagram structure and its transmission pattern, to find the necessity of using it for best practise.\\
\subsubsection{UDP datagram structure}
The UDP datagram structure is defined in the RFC 768\cite{postel1980user}. The UDP protocol is supported by Internet Protocol(IP), and the structure of its package is similar to that of IP packages. It is consisted of two parts: the header containing protocol specific information and the body which carries user contents, showed in the picture .
\begin{figure}[H]
	\centering
	\includegraphics[width=0.5\linewidth]{picture/UDPStructure}
	\caption{User Datagram Protocol package structure}\cite{postel1980user}
	\label{fig:UDPStructure}
\end{figure}
It is easy to find that the header of UDP does not contain the IP addresses, and it introduces a 'pseudo header' containing the routing information included in the IP header. The checksum area is a 16-bit long complement sum of pseudo header, UDP header and the data, however, it can only show if the data is corrupted or not, and cannot tell which bits are wrong.\\
\subsubsection{Important features related to high-frequency data}
The header of UDP is relatively small, therefore the UDP datagrams can carry more amount of user content compared to other forms of network protocols. The checksum mechanism is simple and efficient for ensuring the correctness of received data, however, the latency cost in retransmission is still intolerant for a low-latency network, and engineers have established several mechanisms to reduce the cost, for example, the cached relay, which will keep a cached data of what has been transmitted within certain amount of time, and any retransmission request will be replied by the relay rather than the source if the requested data is still in the cache. Another solution is using two data feeds from the same source instead of one, and it is also widely supported by high-frequency data providers\cite{abraham2002high}.\\ 
\subsection{A-B Data Feed}
A-B feed structure requires two independent signals from one data provider, and the one feed(A) is several milliseconds faster than another feed(B). Relay stations between the sender and receiver need to decompress both feeds, filter and aggregate the package then transmit the data feed to the next station. Receiver also need a decompresser to handle the data from both feeds, reconstruct the original package and split the data into sub-streams forming process-specific subset of data, shows in picture \ref{fig:3}.\\
Each data feed arrives in sequence with a prefixed header, containing information from abstracted layers of network protocols\cite{udpprotocol}. UDP protocol suggests UDP client to provide reliability services that the receiver should perform CRC check of the received package chains, and retransmit the chain if package lost found, time-out mechanism exacerbate the difficulty in providing low latency data service. A-B data feed reduces enable the rapid collect-dispatch system to provide output within 500 milliseconds\cite{zusman1999fault}.\\
The hardware implemented for the high frequency market data are commercialized and deep customized, which is a huge burden for the companies in this field. One of the equipments in the high-frequency market data infrastructure is the dual-feed switches, which accept the A-B data feeds from the network, and translate the feeds into general UDP datagram and make the market data as a service for the users connecting to it.  It will be obliging to find a way of implementing the switch using commodity, or even the open-source hardware.\\
\begin{figure}[H]
	\centering\includegraphics[width=0.7\linewidth]{picture/A-BFeed.PNG}
	\caption{Overview of A-B streaming infrastructure}
	\label{fig:3}
\end{figure}

The dual-feed switch has two features in performance: seamless retransmission of two data feeds, and high throughput capacity in its two ports simultaneously, which is similar to the interconnection network structure of supercomputers. We will continue our discussions with the basic concepts of one particular kind of supercomputers suitable for the commodity hardware, and break down the implementations to produce a possible solution to the requirement mentioned above.\\
\subsection{Beowulf Cluster}
Beowulf cluster is a highly scalable hardware architecture for computer clusters, and very popular in supercomputer design\cite{behrooz2005computer}. The basic principle is providing scalable parallel computational power by scaling number of cores and other resources, the nodes are interconnected in proper designed network topology. This feature makes the Beowulf structure very friendly to commodity hardware, including the personal computers\cite{sterling2002beowulf}. The goal of using a cluster is to implement multi-process programs over isolated cores to improve the parallel performances. Before further breaking down, we need to introduce the concept of computer clusters.\\
\subsubsection{Computer clusters}
A computer cluster is a set of computational nodes consisted of independent Central Processing Units, Memory and other hardware structure\cite{vrenios2002linux}. Each nodes can be an independent, complete computer that can execute certain computation tasks\cite{sterling2002beowulf}. These computational nodes are connected in a certain way, in this thesis particularly, the nodes are connected by Ethernet cables, and using network protocol supported transmission channels to perform internal data exchange. \\
\subsubsection{Constructing a Beowulf cluster}
The key feature in the design of Beowulf cluster is the construction of interconnection network:
\begin{itemize}
	\item Network topology: how the network topology works and related routing strategy
	\item Transmission protocols: we have Internet Protocol family supports implementations like Internet Protocols(IP), User Datagram Protocols(UDP) and Transmission Control Protocols(TCP), their features will have a compelling influence over the cluster efficiency
	\item Abstraction of the intercommunication: applications running on the cluster needs to be isolated from the implementation details of the cluster, and the outputs should be platform independent
\end{itemize}
We put the network hardware aspects down for the moment, and focus on the intercommunication abstraction, because it defines the implementation of transmission protocol over certain type of network topology. It will be a high cost if we choose to implement the communication layer by ourselves, however, there is an established solution and can helps us to construct the abstraction layer with proper configurations. The Message Passing Interface protocol and its implementations are the most popular solution for building the standard communication interfaces for parallel computation\cite{geist1996mpi}.
\subsection{Message Passing Interface(MPI)}
The Message Passing Interface(MPI) is a standard for the message passing libraries established between 1993 and 1994\cite{geist1996mpi}. It was invented aiming at solving the problem of cross-process communications on a distributed system, as the paralleled processes may running on physically independent cores. It abstracts the implementation details of the communication, and makes the framework friendly to heterogeneous system. Computational nodes with MPI support can form a cluster and practise parallel computing ignoring the differences in hardware architecture.\\
MPI has implemented modules to improve the efficiency, including high-level language binding and remote memory access\cite{gropp1996high}. These two features are essential for our solution, and will be discussed in detail in the following section.\\
We can construct a Beowulf cluster with commodity hardware using MPI implementation as the communication layer, and perform concurrent computations over the platform. However, Beowulf structure is not only known for its commodity implementations, some of the world leading supercomputers, including the latest top 1 supercomputer Sunway TaihuLight\cite{fu2016sunway}, and its previous generation MilkyWay-2\cite{liao2014milkyway}. We will have a discussion about the implemented interconnection architecture, and break down to our solutions.\\

\subsection{Field-programmable gate array(FPGA) and Integration with Open Sourced Hardware}
We have introduced several solutions from the view of networking topology and its implement, however, the acceleration for each computational node in the cluster is another important approach. One of the acceleration method is to implement the hardware algorithm module instead of running the software one. The customized hardware can accomplish specific algorithm for certain inputs with better performance. The Field-programmable gate array is the programmable hardware which can perform the hardware algorithm.\\
The introduction of Field-programmable gate array, or the FPGA, aims at solving the heterogeneous computation requirement in hardware based acceleration and general computing by using gateway logic circus to perform general algorithms or general processes\cite{hauck2010reconfigurable}. The FPGA is an integrated circuit, with re-configurable gateway and other logical blocks that can simulate most non-reconfigurable hardware with similar efficiency. The FPGA can be programmed to meet certain functional requirements by certain hardware definition languages, and modern FPGA also supports advances programming language interfaces, for example, C and Python.\\
The integration of FPGA and open sourced hardware have been carried by the community, and supported by the FPGA providers such as Xilin and Spartan. The LOGI-Pi is the very first FPGA supported by the open sourced hardware, with Spartan FX9 FPGA chip and SATA connector directly to the designed hardware board\cite{logipi}.
\subsection{Open Sourced Hardware}
Hardware engineers need to design and print their customized motherboards to support FPGA and other peripherals, however, the industry produces general motherboards supporting most of the ports including PCI-E, HDMI, USB2.0 and USB3.0. Among which the most well-known commodity hardware is Raspberry Pi and BeagleBone Black showed in picture \ref{fig:commodityhardware}. Both are designed as System On a Chip architecture.\\ 
\begin{figure}[H]
	\centering
	\includegraphics[width=0.7\linewidth]{picture/commodityhardware.jpg}
	\caption{Beaglebone Black and Raspberry Pi2 Model B}\cite{bbbrpi}
	\label{fig:commodityhardware}
\end{figure}
\subsection{Scalable Interconnect Switches in Supercomputers}
Existing hardware solutions for low latency network is not suitable for mid sized institutions or individuals to start their usage of the network, and the development in building supercomputers provides an alternative solution to build infrastructure with equivalent function without implementing designed equipments. We will first introduce the acceleration feature in Beowulf cluster, then discuss the fat-tree network topology implemented in the MilkyWay supercomputer, to analyse how it improves the efficiency, and its inspiration for our project.\\
We will also discuss these two topics related: a comparison between the established Ethernet interconnection solution and the hardware direct link solution using Serial Peripheral Interface, and the bottlenecks of the network solution.
\subsubsection{Linear Speed-up in Beowulf Architecture}
The bottleneck now lies in the communication among cores in paralleled program\cite{brown2004engineering}, the bandwidth and latency in interconnect network worsen the efforts in removing the obstacles in improving the performances of the Beowulf cluster.\\
Six kinds of memory are defined according to its logical location in computer, they have distinct latency in read/write and requirements in bandwidth, shows in table \ref{table:1}.\\
\begin{table}[htb]
\begin{center}
	\caption{Latency and bandwidth in memory IO}\cite{johnson2005memory}
	\label{table:1}
	\begin{tabular}{lll}
		\hline
		& Latency              & Bandwidth Requirement \\ \hline
		L1 cache     & 1-5ns(1 clock cycle) & -                     \\
		L2 cache     & 4-10ns               & 400-1000MBps          \\
		Local memory & 40-80ns              & 100-400MBps           \\
		Local disk   & 5-15ms               & 1-80MBps              \\
		NFS disk     & 5-20ms               & 0.5-70MBps            \\
		Network      & 5-50us               & 0.5-100MBps           \\ \hline
	\end{tabular}
\end{center}
\end{table}
We set  $T_{L}$ is the total time cost for a parallel program running in a certain Beowulf cluster, and it is paralleled in n threads. Assume the ideal scenario that n is much smaller than the number of total computational cores N in the cluster, so the threads can run in separate cores without waiting. $P_{L_{i}}$ is the possibility of finding code or data in the ${i}$th type of memory, $t_{i}$ is the IO time cost for ${i}$th type of memory. We can have the expected total time cost in equation \ref{equa:1}:\\
\begin{equation}
\label{equa:1}
E(T_{L})=\sum_{i}^{n}T_{L_{i}}=\sum_{i}^{n}(\sum_{j}^{6}P_{L_{j}}t_{j})
\end{equation}
The time cost is linear related to the increment of IO times in a single core, that rapid data or code IO in one core will increase the time cost significantly.
Instructions are not loaded sequentially in L1 cache for most cases, so the bottleneck of accelerating Beowulf machines lies in reducing the swap among different memories. A simple solution is to break the memory requirements into small blocks, that the block is small enough to be swapped into L1 or L2 cache only once after execution, and runs without halt in content swapping until the result of calculation being transferred to the local memory. Master system dispatches the blocks to subsystems, and it is clear that the total time cost will be approaching to execution time of a single block when the number of subsystem increases.\\

\subsubsection{Interconnect Networks Topology: Perfect Shuffle and Fat Tree}
The Interconnect network in Beowulf clusters is important for acceleration and optimization. Performance of a supercomputer are measured by time cost in route permutations, and cost is defined by number of switching data\cite{leiserson1985fat}. For a parallel computer interconnected by boolean hypercube, the physical volume cost for a $n$ core cluster would be $n^{\frac{3}{2}}$. \\
The design of a layered interconnecting network called 'Shuffled Network' by Schwartz resolved the bottleneck in network volumes\cite{stone1971parallel}. The shuffle follows following rule, for a full set of $N$ indices, each $i$ maps into another set of indices by permutation P: \\
\begin{equation}
\label{equa:2}
P(i)=2i\; \; \; \; 0\leq i \leq \frac{N}{2}-1\\
\end{equation}
\begin{equation}
\label{equa:3}
P(i)=2i+1-N\; \; \; \frac{N}{2}\leq i \leq N-1\\
\end{equation}
and the inverse permutation 
\begin{equation}
\label{equa:4}
P^{-1}(i)=\frac{n}{2}\; \; \; (n\; is\; even)
\end{equation}
\begin{equation}
\label{equa:5}
P^{-1}(i)=\frac{n-1}{2}+\frac{N}{2}\; \; \; (else)
\end{equation}
we introduce the binary representation of data in the original set, that $i=\beta_{D}\beta_{D-1}...\beta_{1}$, so the permutation of $i$ is $P(i)=\beta_{D-1}...\beta_{1}\beta_{D}$, the inverse permutation of $i$ is $P(i)=\beta_{1}\beta_{D}\beta_{D-1}...\beta_{2}$. This defines an ideal switch network that for $o$th step, the data can be divided into even-numbered and odd-numbered processors, and in the $(o+1)$th step, data in even-numbered processors can be moved to low-numbered processors by implementing permutation $P$, and data move from odd-numbered to high-numbered processors can be performed by using inverse permutation $P^{-1}$\cite{schwartz1980ultracomputers}. The arbitrary movements of N elements in shuffle is $O(logN)$, similar to the full hypercube network while decreasing significantly in network volume\cite{clos1953study}.\\
The implementation of a perfect shuffle interconnection is the Fat-tree network. For classic parallel programs, divide and conquer is an efficient solution\cite{aho1974design}. Fat-tree interconnection strategy is a full binary tree. Message routing through children to the parent will carry data from all previous children, thus, the data capacity of the parent $cap(p)=\sum_{chil}^{route}cap(child)$.Data lost in during transmission from $child_{l_{i}}$ to $parent_{i}$ will be detected by the parent node, and retransmission will only occur between the two nodes(if the package still in the cache of the child node). The only exit is the root node, so any message route through the map will have guaranteed reliability with minimum cost.

\subsubsection{Comparison: MPI supported network interconnection and SPI direct data transmission}
Serial Peripheral Interface(SPI) is another commonly used interconnection technology especially between two single chip machines\cite{junger2007method}. It uses direct digital data feed trigger by the chip, and translated into pulse normally to the General-Purpose Input/Output pins, which is a general known hardware ports for pulse signals\cite{lin2008transmission}. The Raspberry Pi board we introduced in the project is equipped with two SPI chips(SPI and i2c over ARM)\cite{jain2014raspberry}. The SPI uses slave-master structure, and abstracted the data transmission process, especially the hardware details in digital-simulate signal translation into high level programming language functions and interfaces.\\
This can be another ideal way to form an interconnected network among the chips, however, we need to compare the SPI technology to the MPI supported Ethernet interconnection.\\
First, the features of SPI data transmission: data in applications can be translated into digital signals and send over GPIO ports shown in picture \ref{fig:GPIOpin}. The transmission do not need to follow any network layered protocols, and the SPI works as a translator between two applications. The speed of bit exchange in SPI is much faster, and much more efficient than the Ethernet protocols. However, using SPI introduces the problem of scalability and expense: as discussed above, the GPIO pins can only transmit one bit on each pin, and we can either choose to send a series of data via one of the pins, or send multiple bits of data in parallel over multiple pins.\\
\begin{figure}[H]
	\centering
	\includegraphics[width=0.7\linewidth]{picture/photo/GPIOpin}
	\caption{26 GPIO pins on the Logi Pi board}
	\label{fig:GPIOpin}
\end{figure}
The former solution slows down the system because even 1Kb data needs to be translated into 1,000 electric pulse with 50MHz system clock, using 20$usecs$. The other solution makes it hard to increase the scale, because nodes in this cluster needs to be able to broadcast, therefore the number of GPIO pins one node needs is $N_{node}\times n_{GPIO}$, $N_{node}$ is the total amount of nodes, and $n_{GPIO}$ is the number of nodes for a bi-directional interconnection between two nodes.\\ 
The MPI supported interconnection Ethernet network, however, do not need to worry about the two problems listed above. The defect of this solution is, it requires 10Gb Ethernet peripherals in the simulated switch cluster, and the equipments is hard to install and purchase.\\
We decided to use the MPI supported solution for the listing reasons:
\begin{itemize}
	\item The SPI solution requires special hardware between each node for synchronization of system clock, and the solution is more focused on the hardware part
	\item We cannot overcome the difficulty in implementing the GPIO port board, as the open-source hardware only supports 40 GPIO pins on the board, however, the minimum number of boards is three. Therefore the SPI interface can only send 20 bit each time between the nodes, and the experiment cannot perform the scalability test as we do not have expandable GPIO pin peripherals
	\item We found a solution to simulate the scenarios in UDP datagram transmission in 10Gb Ethernet network, especially the package loss possibility simulation, over the normal RJ-45 Ethernet network. The solution will be illustrated in the following chapters.
\end{itemize}
\subsection{Discussion:the bottleneck of network performance efficiency}
We have introduced the A-B data feed mechanism,and showed how it reduces the retransmission time especially within the unreliable networks, however, most of the modern super computers do not implement the dual-feed structure in their interconnection networks. It is key to understand the bottleneck of the network performance efficiency, especially in the high frequency data environment.\\
First of all, we need to understand the choice of UDP protocols: people choose the UDP protocols rather than TCP or other reliable transmission protocols, is largely because the headers in datagram defined by RFC is the smallest\cite{udpprotocol}, therefore UDP protocol is one of the most efficient mechanism for data transmission. UDP protocol has three feature that we should notice: the datagrams have checksum in their headers to provide filtering of 'corrupted' data; the UDP receiver stack works as First In First Service, and the datagrams can arrive in any sequence because different routing strategy; the UDP protocol defines how to detect a mistakes, but has not defined processes to correct them, and a full retransmission of the entire UDP packet will be required.\\
Before the next step discussion, we need to highlight the main goal of high performance network solution: reducing the person-to-service(P2S) latency, and the data route between the person to the service on the other side is consisted of: input from human-computer interaction interface or the application intelligent logic, (local system)local memory, (remote system) remote memory, Network Interface Card(NIC), Ethernet infrastructure, including cables, relays, switches and other medium materials, then the data finally reach the NIC card of the server in form of network sockets, and then the application hosted in the server will consume these sockets by invoking the kernel function of slurping contents from the NIC cache.\\
With these processes in mind, we can at least think about two features that will affect the system:
\begin{itemize}
	\item Network latency between nodes along the route
	\item Wait time in the server side
\end{itemize}
The first feature is easy to understand: the network system is sensitive to node-to-node communication efficiency, and the ideal scenario is that a direct physical link only supported by relays is established between the server and user, the bandwidth of this link is one hundred percent owned by the user. In the real world, most high-frequency trading companies have already established this hardware system\cite{budish2015high}. The reason of using UDP datagrams is to reduce the 'meaningless' payload of the transmitted data, because the fewer protocol related bits there are, the more 'effective' loads of information we can put in a fixed length of network sockets. UDP does not have CRC or other Quality of Service(QoS) secure mechanism, which reduces the header significantly compared to TCP or FTP\cite{jacobson1992tcp}. We can reduce the latency in network by improving the reliability of the mediums over the route, or implementing alternative mechanisms at the endpoint, including the A-B data feed,cached relay and High-bandwidth Ultra-low latency network architecture\cite{alizadeh2012less}. The main goal is to reduce the cost in unnecessary aspects.\\
The second feature is decided by the actual logic implementation of the Operating System on the server along with the application on it which provides the services. However, this part is not entirely hardware independent,  as discussed above, the UDP datagram has checksum mechanism as the security performance to ensure the data arrived is expected. Normally, this checksum mechanism is performed by NIC card automatically, and the user level application can only see the datagrams which has already passed the checksum. Therefore, the throughput capacity of a high Input/Output(IO) host is restricted by the implementation of the NIC-OS partnership\cite{livny1997mechanisms}. \\
Therefore, we can break down the listed two feature into the following table:
\begin{table}[H]
	\centering
	\caption{System bottleneck break down}
	\label{my-label}
	\begin{tabular}{@{}ll@{}}
		\toprule
		Bottleneck             & Solution                                                                                                                                         \\ \midrule
		Data efficiency        & \begin{tabular}[c]{@{}l@{}}Using network protocols with simple\\ structure and small headers\end{tabular}                                        \\
		Network reliability    & More reliable materials, e.g. optic fibres                                                                                                       \\
		Network reliablity     & \begin{tabular}[c]{@{}l@{}}Alternative mechanisms, e.g. cached relay\\ to retransmit sockets from the route instead\\ of the source\end{tabular} \\
		End to end performance & \begin{tabular}[c]{@{}l@{}}Larger cache in NIC card, advanced system \\ architecture\end{tabular}                                                \\
		End to end performance & Properly programmed applications                                                                                                                 \\ \bottomrule
	\end{tabular}
\end{table}
\subsection{Similar Solution}
\subsubsection{FPGA accelerated market data feed}
Engineers are not satisfied with the traditional CPU based acceleration solution, and developed an alternative hardware architecture that solves the problem of buffering and filtering high frequency data. This solution introduces direct connection between FPGA chip and network interfaces, the Celoxica AMDC board enables dual feeds for single chip. Compressors and filtering program are burned in the FPGA\cite{morris2009fpga}. The acceleration of stream processing in this solution lies in the customized hardware algorithm in FPGA, and data streams goes directly to the RAM of user cores via DMA message tunnel, the IO frequency of which is higher than that in network interfaces, the structure is shown in picture \ref{fig:4}.\\
However, this solution requires high investments in hardware, especially the motherboard and printed FPGA. Scalability is another weakness of this solution, it does not provide easy solutions for large scale implementations, especially in the scenario of bandwidth upgrading from 1Gb to 10Gb Ethernet. The NIC chip and interfaces need to be redesigned and reimplemented in the printed FPGA chip, and against the principle of System On a Chip, that the algorithm and application should be maintained only by the system supported by the hardware, not the hardware itself\cite{klaas2004system}.\\
\begin{figure}[H]
	\centering\includegraphics[width=0.5\linewidth]{picture/FPGAcall.png}
	\caption{Overview of A-B streaming infrastructure}
	\label{fig:4}
\end{figure}
\subsubsection{TH Express interconnection network}
TH Express interconnection network is the local area net framework for the Chinese Galaxy series super computers. It is a fat-tree topology network, and consisted of two main parts: communication network and monitor/diagnose network\cite{yang2011tianhe}. The former one connects computational nodes, the latter one checks machine status and resolves runtime error.The researchers developed a scalable network interface hardware High-Radix Routing Chip, and using RDMA method to access multiple memory spaces with unified offset, reducing processing time, improve the scalability and increasing the potential bandwidth. The architecture is shown in the picture \ref{fig:tianhe1a}\\
\begin{figure}[H]
	\centering
	\includegraphics[width=0.7\linewidth]{picture/tianhe1A.PNG}
	\caption{Tianhe 1A interconnection network architecture}
	\label{fig:tianhe1a}
\end{figure}
Solution provided by TH Express is efficient but too expensive to achieve, however, it provides an approach to reduce latency in high frequency network data supporting the massive parallel processing system, that scalable commodity hardware supported by RDMA can improve the reliability and bandwidth of the network.\\
\subsubsection{Enyx`s ARM-based System on a Chip solution to market data infrastructure}
Market data hardware infrastructure providers offer variates of solutions to reduce the latency in market data feed. The most recent solution is to implementing FPGA hardware algorithm with ARM based SoC. It accepts A-B feed from normal network interfaces and perform the decompression and filtering via algorithm in FPGA, the ARM CPU focuses on mission dispatch and status monitor.\\
One of the well-known solution provider is the Enyx, the platform of which is based on Altera Stratix V FPGA board. The rebroadcasting latency for the FPGA supported market data processor can reach 1050 nano seconds, and average latency is 1300 nano seconds\cite{ciscoWhitePaper}. Other commercial solutions supports implementation of trading logic in the switch FPGA, that the terminals from the traders only monitors the macro status of the system, and execution of trading signals only relies on the hardware implemented applications, which reduces the event-reaction time significantly.\\ 
\subsubsection{IBM PowerEN solution for ultra-low latency market data}
IBM has provided a system solution using the PowerEN, a combination of network and classic server processor, driven by throughput capacity\cite{pasetto2011ultra}. It is also a System on a Chip solution based PowerPC cores, equipped with four accelerators:
\begin{itemize}
	\item Host Ethernet Accelerator with four 10Gb Ethernet ports and two 1Gb Ethernet ports
	\item XML parsing accelerator
	\item A Best Fit Scheduling Method based pattern matching accelerator for regular expression
	\item A compression/decompression accelerator
	\item A cryptography accelerator implementing encryption, decryption and hashing
\end{itemize}
The solution makes most of the package processing into hardware supported algorithm to achieve best memory efficiency, and increase the throughput capacity by reducing the processing time cost of the network datagrams. The test result is shown in picture \ref{fig:ibmpoweren}. This solution proves that improving the efficiency in network package processing before the data goes to the main memory by using hardware supported algorithm is an acceptable solution for low-latency network.\\ 
\begin{figure}
\centering
\includegraphics[width=0.5\linewidth]{picture/ibmpoweren}
\caption{Messages latency distribution plot on PowerEN}\cite{pasetto2011ultra}
\label{fig:ibmpoweren}
\end{figure}
\subsubsection{Raspberry Pi supercomputer in Southampton University}
The Southampton University has implemented an Raspberry Pi supercomputer with sixty-four Raspberry Pi Model B boards, interconnected several switches and named 'Iridis-Pi'\cite{cox2014iridis}. The performance of the cluster is measured using LINPACK and HPL benchmark, and the single node performance in LINPACK and node increment benchmark by the HPL results are shown in the picture \ref{fig:iridisBenchmark}. The cluster proves that the Raspberry Pi can form an usable parallel computing cluster, even though its power consumption is relatively very low compared to formal parallel computing infrastructures.\\
\begin{figure}[H]
\centering
\includegraphics[width=0.6\linewidth]{picture/iridisBenchmark}
\caption{Benchmark result of LINPACK and HPL on Iridis Pi}\cite{cox2014iridis} 
\label{fig:iridisBenchmark}
\end{figure}
\section{Conclusion}
In this chapter we made introductions to the background knowledge of high-frequency market data and its network infrastructure, especially the A-B data feed it uses to reduce network latency. Then we explained how the scalable interconnect switches are structured in the supercomputers, and focuses on the Beowulf architecture, which is the prototype of our design, the perfect shuffle within the interconnection network and Message Passing Interface. We made a comparison over the Serial Peripheral Interface connection and MPI supported network connection, and explained our choice of the MPI supported one. We also introduced the Field Programmable Gateway Array which holds our hardware algorithm, and two different types of supported open sourced hardware, the BeagleBone Back and Raspberry Pi 2Mode B.\\
Another discussion is made about the bottleneck of the high performance network system. We listed five possible conditions that the latency of a high-frequency system may be sensitive to. Our system design and experiment implementations are established according to these conditions.\\
We also made introductions to two established dual datafeed solutions for market data infrastructure and discussed pros and cons of both solutions. The high performance network switch in TH Express network brings us an new idea about the approach to the dual data feed hardware that a Beowulf structured open source hardware can achieve similar performance over the network with much less cost. The Raspberry Pi cluster in Southampton University proves that the performance of open sourced hardware can be acceptable if they are tuned correctly.\\ 
The following chapter will have detailed introduction of the technologies we implemented.\\ 
\chapter{Implemented Technologies in the Project}
This chapter discussed the details of the four key technology in the project: Remote direct memory access, fat-tree topology network, MPI-CH, and  field-programmable gate array assisted CPU computation. These established technologies are well-known for being able to establish a parallel multi-core cluster. We are going to implement these technologies to build a distributed network switch over open-source hardware, and achieve similar performance as the commercial customized machines.\\
\begin{table}[H]
\centering
\caption{Implemented Technologies}
\label{my-label}
\begin{tabular}{@{}l|ll@{}}
\toprule
Technology                                                               & Implementation                                                          & Details                                                                                                                                                          \\ \midrule
\begin{tabular}[c]{@{}l@{}}Remote Direct \\ Memory Access\end{tabular}   & Infiniband RMDA                                                         & \begin{tabular}[c]{@{}l@{}}Infiniband verb implementation\\ over RDMA data engine and \\ protocols\end{tabular}                                                  \\ \hline
\begin{tabular}[c]{@{}l@{}}Fat-tree topology \\ netowork\end{tabular}    & \begin{tabular}[c]{@{}l@{}}Fat-tree over\\ infiniband RDMA\end{tabular} & \begin{tabular}[c]{@{}l@{}}Fat-tree network for RDMA \\ internet layer to communicate\\ more efficiently\end{tabular}                                            \\ \hline
\begin{tabular}[c]{@{}l@{}}Message Passing\\ Interfaces\end{tabular}     & MPI-CH standard v2                                                      & \begin{tabular}[c]{@{}l@{}}Integrating the hardware platform \\ to a multi-process cluster over a \\ shared memory managed by RDMA\end{tabular}                  \\ \hline
\begin{tabular}[c]{@{}l@{}}Field Programmable\\ Gate Array\end{tabular} & Sparta FPGA                                                             & \begin{tabular}[c]{@{}l@{}}Introducing FPGA to assist CPU \\ clusters handling problems that \\ are hard to be separated into \\ paralleled threads\end{tabular} \\ \bottomrule
\end{tabular}
\end{table}
The implementation of these methods are carefully selected, to fit the architecture and achieve the best performance. MPI structure with RDMA is not a new technology, however, it has not been implemented over the dual-feed switches yet, so the design of the structure is different from a normal MPI cluster. The network topology in implementation is also another innovation because this design is usually in the computational nodes, and the project introduces fat-tree RDMA network to reduce the requirement in hardware performances of single node.\\
\section{Remote Direct Memory Access(RDMA)}
The introduction of remote direct memory access technology is inspired by the architecture design in TH Express high performance interconnection network switches, which implements the RDMA as a basic memory space management infrastructure to enable the scalability of the system using distributed cluster as an virtual switch rather than using customized central switch. The basic principle of using RDMA is to trade for efficiency with the cost of memory space and bandwidth within the switch system\cite{pinkerton2002case}.\\ 
Remote direct memory access technology is developed over the traditional Direct Memory Access engine, filling the requirement of encapsulating details in data copying and shifting in cloud platforms\cite{archer2012remote}. The technology enables developers ignoring the implementation in data transmission among computation nodes, and focusing on the design of the parallel computing algorithms themselves. The actual RDMA protocol family is consisted of three protocols:
\begin{itemize}
	\item Remote Direct Memory Protocol
	\item Direct Data Placement Protocol
	\item Maker UDP Aligned
\end{itemize}
The protocol family focuses on providing data availability over the interconnected networks without operating system kernel involved, and direct data exchange in the network interfaces. The technology requires no additional buffering spaces and performs under atomic operations: read/write, send/receive\cite{RobertRDMAintro}. \\
The problem of supporting remote direct memory access is the implementation of user-level data access over traditional network protocols, which would inevitably require the interception from system kernel model\cite{liu2004high}. Remote DMA requires interception before data swapping and access in the physical memory, therefore interception in kernel model must be implemented. However, the feature required by applications that RDMA should hide the details of data shifting and other system-level details, and application structures should not be infected by the implementation. For example, data shifting within the multi-processing distribution system should not infect the processes of single threads, which may require virtual data blocks whose actual address is on the remote node. Another common scenario is the asynchronous exiting sequence of different threads, which requires asynchronous data shifting in entering and existing actual memory spaces. \\
Engineers have introduced several features in the implementation of RDMA protocols, that bypasses the kernel model in the traditional internet protocol layers, and created a virtual tunnel among buffers in the threads running on separate nodes in the cluster.\\
RDMA plays as a core part in the project, which enables the open sourced hardware performs integrated computation with incoming data feeds and share without CPU involvement,  minimizing the computational power cost by increasing the bandwidth and memory space cost, which is far easier and cheaper to achieve than high performance processors.\\
\subsection{Two Important Concepts: Verbs and Queue Pairs}
The implementation of verbs and queue pairs enables the RDMA performing kernel bypassing and CPU-free message handling\cite{recio2003rdma}. The verbs are abstract APIs that can be called in the application layer indicates that the application is implementing RDMA thus the layers below, especially the traditional socket-based internet protocol layers will be covered and the network interface card should be ready to be turned into remote memory direct access enabled model.\\
\begin{figure}[H]
	\centering
	\includegraphics[width=0.6\linewidth]{picture/rnic.PNG}
	\caption{RNIC Model Overview}
	\label{fig:rnic}
\end{figure}
In this model, the system will handle UDP packages with DDPP header different from other packages that it will be processed by the NIC card and sent to local memory directly with RDM protocol management. The queue pairs can be viewed as the schedule handler that performing memory registry and calling the RDMA enabled NIC, or the host channel adapter(HCA) to caring pinning of the memory packages\cite{subramoni2009rdma}. The basic model overview is shown in picture \ref{fig:rnic}\\
\subsection{Queue Pair}
The queue pair model is based on the consumer-provider model, which handling queuing of the tasks in an asynchronous way, and implements the actual data models in the network interface level to support the verb functions\cite{garcia2006binding}. The queue pair solves the problems of:
\begin{itemize}
\item Asynchronous arrival of data blocks in the interconnected network
\item Fault detection in data transmission
\end{itemize}
The queue pair is an integrated system of software, hardware and framework in remote network interface card(RNIC), and the implementation of RMDA protocol family is in the RNIC framework. The protocol family can be divided into three levels: the \(l2\) level Ethernet access mechanism implementation; the \(l3\) level IP protocol layer implementation, including IPv4, IPv6 and IPSec; the \(l4\) level protocols for TOE over TCP protocols, and RMDA protocol family including RDMA, DDP and MPA\cite{arndt2003infiniband}. RNIC infrastructure enables TCP transport support for TCP packages bypassing system kernels, which intercepts data exchange from NIC to memory. \\ 
\begin{figure}[H]
	\centering
    \includegraphics[width=0.6\linewidth]{picture/queuepair.JPG}
    \caption{Queued pair communication over internet}
    \label{fig:queuedpair}
\end{figure}
\subsection{RNIC Verbs}
The RNIC verbs describes the behaviour of RNIC infrastructure, and creating programming interfaces for applications to make use of the infrastructures\cite{hilland2003rdma}.  The RDMA is built upon a consumer-provider model, and the verb methods act as the middle ware between the application consumers and actual queuing pairs inside the RNIC software logics\cite{krause2008method}.\\ 
The verb specifications defines functions and semantics the application need to access the RDMA protocol layers, however, the verbs only include connection management and tear-down semantics, and availability to additional protocol layers  are not defined in the verbs.\\ 
\subsection{Implementation Details of Verbs}
The implementation of RNIC verbs are related to the actual policy defined by the queue pair politics, and different in actual infrastructure environment. In this project, the RDMA is the memory accessing system which provides buffering services for dynamic hardware structure for high performance system. \\
\subsubsection{Memory management}
RNIC introduces two concepts in memory management: memory window and queued pairs. The memory window manages the organization of contents in each node, and queued pairs defines the way separate nodes communicating with each other\cite{krause2008method}.\\
The memory windows are the basic swap element between the actual memory block and RNIC buffer\cite{garcia2006binding}. The RDMA introduces memory windows mechanism to perform memory registration and management. The remote memory shifting and offset table management are implemented through memory windows. \\
Memory space management are performed through RNIC data drivers via Tagged Offset\cite{boyd2007memory}. From the application consumer view, the memory spaces are continuous and the application can visit the entire 'shared' space by applying simple offset, but it works different from the queue pair view.\\
The queue pair is the mechanism which work requests are translated by the verbs into the interfaces provided by RNIC library, and be ready to enter the implementation of the data engine before going into the actual internet protocols. The working queues are maintained by separate nodes, and perform as the basic element of content swapping instructions. The swapping of the memory content, instructed by working queue I/O, is indexed by queue element ID and memory window pointers\cite{hausauer2006rdma}:\\
\begin{algorithm}[H]
	\KwData{ID of queued process}
	\KwResult{Swapping contents between two remote processes}
	initialization\;
	\While{Working queue is not null}{
		get memeory window pointer by queued process ID\;
		get destination process ID of the paired queue\;
		\eIf{destination found}{
			pop out the send queue\;
			swap memory content instructured by window pointer\;
		}{
		throw exceptions to user\;
	}
}
\caption{Content swapping between two memory blocks}
\end{algorithm}
So the problem lies in the strategy in choosing the strategy in the cross-node STag management. We can choose either using the same queued process id for all queued pairs, and client can seamlessly access the process registered with the same id in the server, or using different process id on each queued pairs, and managed by the client-server communication\cite{hilland2010method}.\\
Both ways have problem: for the first solution, multiple clients can access each other`s process for they enjoy the same id in the server, which is public accessible; the second solution cannot fulfil the requirement of unified memory window management in multi-client environment, which servers need register multiple memory windows for separate clients.\\
The solution provided by RNIC is a combination of both solution:
\begin{itemize}
	\item At binding phase: queued pair id binds with each memory window, and different in separate nodes;
    \item At accessing phase: memory window with QPID is registered in both source and destination paired queues with same QPID;
\end{itemize}
\section{Fat-tree topology implementation over RDMA}
\subsection{Reducing the Cost in Node Intercommunication}
Cost in node intercommunication lies in to aspects: the content transmission speed between the network interfaces and memory spaces, and the latency in data links among nodes\cite{arimilli2010percs}.\\
Data swapping within RDMA system is a bandwidth consuming performance, and the efficiency in data transformation is the threshold for the overall efficiency of the system. \\
In the data engine level, NIC can access the data blocks with the direct memory addresses without CPU, and improving the accessing speed of data by direct I/O performances from NIC to memory. \\
\\
\begin{algorithm}[H]
	\KwData{memory pointers of source and destination}
	\SetKwProg{Fn}{Function}{ is}{end}
	get destination process id from the popped queue\;
	find memory pointer of the destination process\;
	\eIf{found}{
		calculate the offset between source and destination input\;
		real memory pointer = found pointer + offset\;
		}{
			throw exception to user
		}
	\SetKwFunction{KwFn}{Fn}
\caption{Direct Data access from NIC to memory}
\end{algorithm}
In the network topology level, the RDMA network protocols introduces fat-tree topology, which 'parent' nodes have the bandwidth aggregating bandwidth of all children nodes, and the perfect shifting mechanism guarantees the minimum cost in data shifting.\\
The implementation of perfect shifting is based on one assumption: most of instructions performed by applications require data from adjacent memory blocks $T_{i+1}(d)=Addr_{i}+V(T_{i}(d))$, therefore, the 'next step' data requirement process always requiring data either:
\begin{itemize}
	\item Require data from local memory space
    \item Require data from the neighbour nodes with the same parents
\end{itemize}
The implementation of dual feeds in data sources also reduces the time cost significantly, for the bypassing of kernel model in UDP package processing, the client in the network interface in switches works different in error processing, and node level error can be detected in leaf level switch\cite{gu2009low}. Comparing to flat structured centralized switch, the fat-tree topology interconnection reduces the expected error detection time from $E(T_{t}|e)=E(T_{node}|e)+E(T_{route})$, however, the expected time of processing error in node($T_(node)$) is a constant, assuming the cluster is homogeneous, and reducing the error processing time in network transmission phase.\\
\subsection{Analysing flat topology and fat-tree topology}
\begin{figure}[H]
	\centering
    \includegraphics[width=0.4\linewidth]{picture/FlatStruct.PNG}
    \caption{Flat structured network}
    \label{fig:flat}
\end{figure}
The nodes can be divided into three groups according to the topology position to each other:
\begin{itemize}
	\item Adjacent nodes: nodes have the same parent node
    \item Non-adjacent nodes in the same field: nodes have different parent nodes, but the nodes have direct interconnection
    \item Non-adjacent nodes in different field: nodes have different parent nodes, and the nodes have no direct interconnection
\end{itemize}
As shown in picture \ref{fig:flat}, the $N_{i}$nodes can detect and process errors in transmission and calculation, and $S_{j}$ works as the network switches which cannot process data errors. The average processing time from node N1 to adjacent nodes in network can be divided into three parts:\\
% Please add the following required packages to your document preamble:
% \usepackage{booktabs}
\begin{table}[H]
\centering
\caption{Average processing time for node N1}
\label{tab:n1proc}
\begin{tabular}{@{}ll@{}}
\toprule
Equation                                                                                          & Comment                                       \\ \midrule
$E(T_{adj})=T_{\widehat{N_{1}S_{1}N_{i}}}=\frac{\sum_{1}^{n}T_{i}}{n}$                            & i is the adjacent node                        \\
$E(T_{\widehat{N_{1}S_{1}{S}'_{1}N_{j}}})=2\times E(T_{adj})+2\times T_{\widehat{S_{1}{S}'_{i}}}$ & j is the non-adjacent node in different field  \\
$E(T_{\widehat{N_{1}S{N}'_{k}}})=2\times E(T_{adj})+E(T_{\widehat{S_{1}{S}'_{i}}})$          & k is the non-adjacent node in the same field \\ \bottomrule
\end{tabular}
\end{table}
From the equations it is easy to observe that the most time-consume process is the error detection when data is route through different fields, however, this could be frequent even we assume that the instructions in most cases process data in the adjacent node, because the switch nodes in different nodes in the node cluster, for example, $N_{1}$ and ${N}'_{1}$ could be logically adjacent in the computational cluster it connects to.\\
There are two options to reduce data processing time in the switch cluster shown in the table \ref{tab:n1proc}, one is to make all nodes adjacent to each other, however, the structure of the centralized hub could be over-complicated, another one is to to connect all the fields together,the problem is the complexity in constructing the interconnection network. A full mapping among the field hubs requires each hub supporting bandwidth of $bandwidth_{n}\times n_{fields}$, which can be very costing if we need to implement the high bandwidth in each field node.\\
One possible solution to avoid the increment of bandwidth requirement is to implement layered interconnection network rather than a flat structure, which resolve the bandwidth requirement to the back bone layer rather than the entire communication network. \\
\begin{figure}[H]
	\centering
    \includegraphics[width=0.5\linewidth]{picture/flat_addNode.JPG}
    \caption{Adding one node in flat mapping in the same field, six connections are changed}
    \label{fig:flatadd}
\end{figure}
The implementation of a tree-structure properly solves the problem, which not only provides a layered infrastructure which implements an aggregating bandwidth requirement from leaves to the root upstream, but also ensures the scalability of the system at the minimum cost, and the binary tree search time is $log(N)$ \cite{ellis1980concurrent}, because the increment in leaves only affects the sub-nodes in certain fields, rather than the entire system.\\
\begin{figure}[H]
	\centering
    \includegraphics[width=0.6\linewidth]{picture/tree_addNodes.jpg}
    \caption{Adding nodes in tree structure, only two interconnections are changed}
    \label{fig:addnode}
\end{figure}
\subsection{Rebalancing the fat-tree over network bandwidth}
Another benefit of introducing the tree topology is the quick rebalancing feature of a tree topology network. The unbalanced tree will cause an unbalanced bandwidth load in the system. Advanced algorithm can rebalance the tree from a vine to a balanced binary tree with the time complexity of$O(n)$, and the amount of the nodes only have linear affect on the rebalancing practises. The extreme situation is a $m$ depth vine, whose root only have one leaf in one side, and another vine with $m-1$ depth vine.\\  
First, introducing the compressing of a vine in the rebalancing:\\
\\
\begin{algorithm}[H]
	\KwData{the pointer to root node, and the depth counter}
	\KwResult{A compressed tree}
	point the scanner to the root\;
	\For{$i\leftarrow$ 1 \KwTo counter}{
			$child\leftarrow$ scanner.right\;
			$scanner.right \leftarrow$ child.right\;
			$child.right \leftarrow$ scanner.left\;
			$scanner.left \leftarrow$ child\;
		}
	\caption{Compression of an unbalanced tree}
\end{algorithm}
and the steps from $k$th compression to $(k+1)$th compression can be shown as the figure below:
\begin{figure}[H]
	\centering
    \includegraphics[width=0.6\linewidth]{picture/compression.JPG}
    \caption{Compression step k to step k+1}
    \label{fig:compression}
\end{figure}
The compression process will be performed each turn in rebalancing, for an vine with $n$ depth, each step will make a compression of the rest $\frac{n}{2}$ branches, and the node of the vine can be a complete binary tree, take the example showed in picture \ref{fig:compression}, nodes 2,4,6,8,9 are all complete binary trees, and the compression will not change the structure of the binary threes, because the only the vine nodes 1,3,5,7 are changed by the compression process\cite{stout1986tree}.\\
The rebalancing of the tree structure resolves the balancing problem of interconnection bandwidth, take the bandwidth requirement of node 5 for example: in step k before rebalancing, the bandwidth requirement of node 5 on the left side is $b_{1}$ for only one node, and on the right side is $\sum_{m}b_{i}+1$, $m$ is the amount of the leaves after vine node 5. After rebalancing, the bandwidth of node 5 on each side is the same($2\times B_{b}$, $B_{b}$ is the bandwidth of a balanced tree).\\ 
The introduction of rebalancing tree in the network topology of fat tree is for the stable scalability, and new nodes adding to the system will not have to be added following certain topology rules: the balancing of bandwidth requirement are maintained by the master node automatically, and the topology of the system is maintained dynamically.\\
\section{RDMA verb implementation: Infiniband Verbs}
Infiniband verbs, or the iverbs, are the verbs supported by infiniband RDMA protocol family which supports the implementation of RDMA over general hardware clusters\cite{mitchell2013using}. The build-in network topology management can be configured into fat-tree, which is discussed above that we want to introduce in the system. \\
The Infiniband provides upper-layer protocols(ULPs) supporting not only RDMA itself, but also most features in MPIs\cite{liu2003performance}. The reason not using the entire Infiniband protocol family is that the support of message passing interfaces in infiniband is ill-supported in scalability, and the version we use for multi-processor interface for the open-source hardware cannot perform ideally in the infiniband pre-configuration, because the passing interfaces designed in the iverb is for much heavier use, including distributed service providers and computation. Our system is a pure message passing system which only filtering and passing message for the clusters, rather than perform the computation for them.\\
The iverb implementation can be divided into two parts\cite{bedeir2010building}. The client part and the server part. Server side mainly listens to the sockets, and react to the sockets send from the client side. The client send sockets to the servers in order to fetch data from memory directly via RDMA mechanism\cite{jiang2004efficient}.\\ 
The iverbs lie in the client side when the application need to communicate with the server(the details are hidden by the verbs) to fetch data and instructions, and the server side when request are queued in the memory spaces, verbs will translate the queues to working queue elements and find the memory windows it points to, reply with the correct memory space addresses to the client to fetch the data\cite{santhanaraman2008designing}.\\
Two features must be guaranteed for the full functioning of iverb system:
\begin{itemize}
	\item Asynchronous reliable communication: both sides will receive hanging notifications and instructions 
    \item No buffers in application level: the synchronization of the send-receive system requires no buffering especially in application memory spaces, which will cause serious problems if the content of application data is modified after sending without completing.
\end{itemize}
\subsection{Building the Passive Side}
The passive side creates and maintains an event handler for queue pairs and an event listener running in the background of network interfaces, which listens to the events from the active sides and send request queues to the queue pairs in the queue event handler directly without implementation of CPU\cite{kashyap2006queue}. The iverbs in the event handler will then send the response including the memory content to lower-level protocols to be packaged into sockets, and the network interfaces will then send the package to the destinations.\\
\begin{figure}[H]
	\centering
    \includegraphics[width=0.5\linewidth]{picture/passive.JPG}
    \caption{Passive side design}
    \label{fig:passive}
\end{figure}
\subsubsection{Building the Active Side}
The active side runs in the client, which directly queries the content from the 'virtual' memory spaces, which is managed by the server. The active side runs a queued event handler which consumes working queue elements translated by the iverbs from the applications, and resolves receiving queues from the server, getting the actual content of the shared memory spaces\cite{kashyap2006queue}.\\
The active side maintains a route to peer nodes for both active and passive sides, in order to communicate among the clusters for swapping data and instructions.\\
\begin{figure}[H]
	\centering
    \includegraphics[width=0.8\linewidth]{picture/active.JPG}
    \caption{Active side design}
    \label{fig:active}
\end{figure}
\subsection{Conclusion}
This section introduces the RDMA method for memory management, and the fat-tree topology we need to implemented assisting the RDMA for better performances.\\
The purpose of implementing RDMA in the project is to overcome the weakness in computational power in open-source hardware, which is usually based on ARM processors with limited memory spaces. The established commercial solution to this problem usually contains an unified memory flash.The RDMA enables us to introduce fragmented memory hardware into an virtually unified one. \\
The purpose of implementing fat-tree topology is inspired by the TH-Express high performance computation clusters, which uses fat-tree interconnection network to solve the problem of low-latency network infrastructure\cite{pang2014th}. In this particular project, fragmented nodes makes it different than implementing a centralized routing system, which have fixed bandwidth balance. The implementation of rebalancing load among node clusters makes it possible to dynamically add/remove nodes.\\
\section{MPI-CH}
Researchers introduce the MPI to build a portable and scalable multi-threading platform over heterogeneous hardware environment\cite{dongarra1995introduction}.To understand the necessity of this interface, a brief introduction of message passing model need to be introduced.\\
Most parallel applications were for scientific purposes\cite{kendall_2016}. Most of the libraries for such applications uses a message passing model for exchanging data among different cores of the machine, and normally a master manages multiple slaves by dispatching tasks and gather results.\\
The MPI-CH is an implementation of the message passing interface standard, and is widely used over multi-processing machines as a data exchange model over multiple processors performing multi-threading computations. It covers almost every aspect of MPI standards, and widely supports far more different hardware architectures compared to another well-known implementation, the OpenMPI.\\
The MPI-CH is the solution for implementing scalable dual-feed receiver in this project, to enhance the efficiency of communication among the nodes using a message passing interface, instead of building a complicate interconnection structure from ground zero. The project makes use of the listing features from MPI-CH compare to other solutions:
\begin{itemize}
	\item Easy to build and configure message passing interfaces
	\item Remote memory access mechanism to enable the scalability
	\item Fat-tree topology to balance the efficiency in hardware architecture of single nodes and network bandwidth
\end{itemize} 
\subsection{Introduction to SPMD and MPMD}
A parallel application can be divided into two different kinds according to the way it uses data and instructions, Single Program Multiple Data and Multiple Program Multiple Data.\\
Before introducing the SPMD and MPMD,it is necessary to make a brief introduction to the concept of Single Instruction Multiple Data(SIMD) and Multiple Instruction Multiple Data(MIMD).\\
SIMD is the model that each node execute same machine instructions over different data, for example, for array ${A}$, executing $A_{i}+1$ on each element of ${A}$ can be performed by a SIMD process by implementing the ${plus 1}$ method on each paralleled cores. MIMD works in a different way: for calculation of equation $A+B+C-D+E\times F$ , an MIMD implementation can divide the calculation into $(A+B)_{node1}+(C-D)_{node2}+(E\times F)_{node3}$ to three nodes with different instructions.
Both SPMD and MPMD are the subsets of MIMD\cite{darema2001spmd}. In this project, MPI-CH library supports both SPMD and MPMD mode, the system can differ the model by using the command line tool indicating how the program and data being seprated.\\
\subsubsection{Single Program Multiple Data}
Single Program Multiple Data(SPMD) is the parallel application which only implements one application, and execute it among each node with different sets of data.\\
A good example of message passing model is a shellsort process: for an $l$ length array $A_{l}$, the shellsort process divides the array into $\frac{l}{mod(\frac{l}{2})}$ fragments, and perform quicksort process on each fragment.The quick way to improve the efficiency of the algorithm is to place each quicksort process on a independent computational core, therefore, the time complexity only depends on the strategy of choosing the length of steps $mod(\frac{l}{2})$. However, it leaves a problem of communication among the cores.\\
For step $k$, the algorithm uses $n$ amount of cores, and in step $k+1$, the content in the memory space of each core will have the following two steps:
\begin{itemize}
	\item{1.} Merging together into a new array ${A}'$
    \item{2.} Dividing ${A}'$ into $\frac{n}{2}$ fragments and dispatching them to $\frac{n}{2}$ cores
\end{itemize}
Therefore, the programmer only need to design the dispatching mechanism and quicksort algorithm in each core. However, the solution could be cumbersome to achieve without MPI model due to the fact that for each step, the number of cores need to wake and perform send/receive practises will be different, and the designer need to predict the number of cores in each step. However, the time cost in communication is very high as the centralized node need to communicate to each node two times each step.\\
\subsubsection{Multiple Program Multiple Data}
Multiple Program Multiple Data performs parallel practises in a different way. Take the shellsort for example, if the parallel application is running on a heterogeneous cluster, whose node has different software and hardware structure and the programmer need to design and implement different sorting algorithm on each node. Therefore the step $k$ to $k+1$ could be:
\begin{itemize}
	\item{1.} Merging the fragments into new array ${A}'$
    \item{2.} Querying the infrastructure library and forming the dispatching strategy for step $k+1$, including the number of nodes, structure of the implemented nodes and data requirements for each node
    \item{3.} Implementing $k+1$ step, and divide the array ${A}'$ into required fragments, transmit the data into the destination
\end{itemize}
The MPMD program is hard to design due to the difficulty in managing the data fragmentation process, which differs according to the configuration of each node.\\
MPI-CH implements both SPMD and MPMD models, and the host can differs the structure by using command line to submit tasks from master to slave nodes.
\subsection{The architecture of MPI-CH}
The core part of MPI implementation is the send-receive pair between processes, forming a programming interface following certain message protocols with semantic specifications over varieties of machine implementations.\\
The architecture of MPI-CH standard can be abstracted as the picture below:
\begin{figure}[H]
\centering
\includegraphics[width=0.6\linewidth]{picture/MPIarchitecture}
\caption{MPI architecture}
\label{fig:MPIarchitecture}
\end{figure}
The design for the MPI-CH infrastructure follows the following principles\cite{gropp1996high}:
\begin{itemize}
	\item Sharing code without compromising in performance over the actual implementation
	\item The MPI-CH should be easy to dock on different platforms with minimized amount of code changing
\end{itemize}
The first principle is to meet the need of re-implementation lower-level objects in MPI implementations over opaque objects, including communicators with tags on it, because most of these contents in MPI-CH is platform independent, and code sharing will solve the problem of dynamic scalability. The second principle requires modularised design of the infrastructure, which can replace the platform independent modules, especially the sharing part, easily and dock to the platform with minimum cost.\\
The implementation of MPICH can be split into three levels according to the sequence from application to the actual hardware.\\
\subsubsection{MPI API}
The MPI reduce, or the MPI API for the applications, are the highest layer which implements user-level abstract objects and the user applications can introduce these elements to get access to MPI-CH managed resouces.\\
The high portability feature of MPI-CH makes the interfaces easy to use and independent to actual hardware. One application can run on different MPI-CH supported hardware clusters by using configuration parameters. The entire application can be wrapped by two functions: MPI\_Init(num\_arg,*vec\_args) and MPI\_Finalize(). The program can perform both SPMD and MPMD inside the curve, and the MPI standard does not define the behaviour inside body\cite{mpich_doc}.\\  
\subsubsection{Communication mode in MPI}
It is essential to introduce the communication mode defined in the MPI standard before moving to the actual implementation. The MPI standard defines four communication mode: \textit{Standard}, \textit{Synchronous}, \textit{Buffered} and \textit{Ready} modes\cite{dimitrov1999efficient}. It is easy to understand that RDMA always need to transfer small amount of data in high frequency over the network, therefore the protocol strategy cannot be too complex. The RDMA introduces rendezvous communication protocol to perform the transmission.\\
\begin{figure}[H]
\centering
\includegraphics[width=0.6\linewidth]{picture/rendezvousMode}
\caption{Rendezvous communication in RDMA}
\label{fig:rendezvousMode}
\end{figure}
This zero-copy rendezvous protocol in RDMA works in the following sequence:
\begin{itemize}
	\item[1.] The buffer address is wrapped into control messages and transmitted to the receiver
	\item[2.] The receiver decodes control message and find the address in the receiver`s buffer, and send acknowledge message to the sender
	\item[3.] The sender sends data to receiver, and send finish segment when the transmission is finished
\end{itemize}
It uses eager strategy in pushing data to the receiver, however, there are still problems in this send-receiver strategy, especially when MPI implements the RDMA feature. The problem will be discussed afterwards.
\subsubsection{Abstract Device Interface}
The MPI-CH implements the mechanism Abstract Device Interface(ADI) in order to provide abstract services for the upper level functions, and hide the details of hardware implementations to the applications\cite{gropp1994abstract}. The message passing happens in this layer when the application uses MPI\_send or MPI\_receive, which indexed by the \textit{handle}, and queued in a way to ensure the execution sequence. The queue is managed by the MPI-CH directly and abstracted from the hardware implementation, and the handlers of IO behaviours are invoked by the lower layer, which slurp the contents and package them into proper format to suit the requirement of different mediums. The descriptions in handlers defines the type of interfaces it needs to be invoked.\\
The idea for the ADI design is to abstract the communications among nodes, and parallel scheduled from the view of applications should have no difference between local cores and remote ones. This due to the fact that MPI standard defines only the local performances in the first version of standard, and the development in network technology enables the hardware designer to introduce remote resources into a bundled parallel computing structure\cite{liu2004high}. 
ADI contains three queues: \textit{send\_queue}, \textit{posted\_recv} and \textit{unexpected\_recv}.
Send queue is for the outgoing messages, and the other two are for receiving messages. The reason for dual queues for receiving is for the asynchronous communication. Application can have the two scenarios when receiving the message from other nodes in MPICH:
\begin{itemize}
	\item The application is ready for receiving a message, and send \textit{MPI\_recv} request to the ADI layer via the API
	\item The application is not ready for any new messages, however the ADI layer receives a message with the handler pointing at this node
\end{itemize}
The first one will trigger an content shift from the queue manager to the application buffer with required contents in the queue, and the second one will register a description in the MPICH runtime manager, which enables that when the application calls for the content from receiving queue it can be shifted to the user buffer directly. \\
Another queue the ADI manages is the device ranking queue, and this queue is initialized by the user when the distributed application is deployed. It contains all the devices the MPICH runtime manager have access to, and abstract the invoker as an communication interface. The application do not need to know the existence of this list, and by invoking the communication interface, runtime manager will post the messages according to the pre-configured ranking list after an outgoing handler is created. When receiving the queue, runtime system will fairly threat all the resources, and no priority in receiving queues. The ADI uses an round-robin style of circulating query around the devices and asks if there is a new message\cite{protopopov2001multithreaded}.\\

\subsubsection{RDMA channel implementated on ADI3}
MPICH introduces ADI3, the Abstract Device Interface version three, which enables programmers to implement their own communication strategy among nodes. The structure of ADI is consisted of two parts, and can be presented in the following fashion:
\begin{figure}[H]
\centering
\includegraphics[width=0.5\linewidth]{picture/ADI3}
\caption{Structure of ADI3}
\label{fig:ADI3}
\end{figure}
The CH3 is a bundle of functions which implements the interfaces for different format of communications, including TCP socket and SHEMEM channel. The reason to abstract these interfaces in a subclass of ADI3 is that the CH3 bundles them into \textit{channels}\cite{mpich_doc_adi3}. The channels are based on normal Unix based socket, and introduces features such as queuing and listeners to achieve better performances.\\
The implementation of RDMA over MPI-CH structure is introduced via this mechanism, because the ADI is consisted of a set of macros and functions which allows varieties of different software-hardware infrastructure to be implemented. The design of this mechanism allows the connected hardware to implement their own message queues and data processors, which is required by the RDMA verb mechanism which overrides the kernel model in UDP socket processing and memory management by introducing its own network data exchange protocols. The RDMA channel only contains five functions, and will be discussed in the next subsection.\\
\subsubsection{RDMA channel implementation}
RDMA channel is designed for global memory share mechanism. It contains two functions for communication, \textit{put} and \textit{get}. Other three functions are for process management. Shown in the discussion of RDMA above, it maintains a queue between two nodes separately.\\
Both \textit{get} and \textit{put} are non-blocking, and the data they manipulate can go directly into the queue rather than wait until the entire process to finish. It is easy to see that the actual implementation of these two functions are different from RDMA standard description, whose communication methods are one-sided, but the actual implementation in CH3 is two-sided.\\
\begin{figure}[H]
	\centering
	\includegraphics[width=0.5\linewidth]{picture/senderreceiver}
	\caption{The message queue between nodes}
	\label{fig:senderreceiver}
\end{figure}
The put-get pair can manipulate the buffer in a straightforward fashion, and the memory buffer can be viewed as an circle: the pair holds an fixed size buffer with an fixed head index, and the put method can insert contents into the circle start from the head buffer, meanwhile the get method invoked by the receiver can read contents also start from the head index and points at the tail. Therefore, the buffer\_read method, which is time consuming, can be asynchronous with the buffer\_write, as long as the buffer is not full. This way of communication also makes it easy to synchronize the memory addresses between the sender and receiver, because the communication buffer uses only one header index and both put and get starts from the same place, only with the opposite direction in the circle.\\
The reason of implementing an independent RDMA channel over CH3 rather than creating a new subset of communication protocols of ADI3 is:
\begin{itemize}
	\item Modern implementations already have similar shared memory mechanism in CH3 layer, and the improvements in a subset of CH3 can inherit the interfaces of CH3 itself, and other system built purely on CH3 can benefit from it
	\item Potentially, other MPI mechanisms will be introduced into the MPICH implementation, for example the Infiniband RDMA which discussed above that it has one-side communication method, and the new implemented methods can use the RDMA channels by following the CH3 interfaces
	\item The portability requirement of MPICH design makes it necessary to consider a general design of communication channel rather than a specific one
\end{itemize} 
\subsubsection{The fat-tree topology}
In this project, the RDMA channel performs as an optimized data exchange channel for the received data, however, the CH3 does not define the way RDMA channel organizes the network topology.\\
As discussed in the RDMA section, the best practise for RDMA performance is to implement fat-tree topology within the cluster, especially under the circumstance that nodes are not powerful enough to process complex computation tasks, therefore the design of software implementation over the cluster needs to sacrifice the bandwidth resources to achieve better overall performance with ARM based CPUs.\\
The fat-tree topology pairs the dual-feed receivers as leaves of the bottom layer, along with an upper layer node also supported by ARM CPU, the added bandwidth among the three nodes is equal to the output.
\begin{figure}[H]
\centering
\includegraphics[width=0.7\linewidth]{picture/fattreeimpl}
\caption{Fat tree implementation}
\label{fig:fattreeimpl}
\end{figure}
We expect the RDMA in MPICH can reduce the requirements computational power of each receiver, and the parent node $node_{3}$ can have better hardware support to perform more complex algorithms compared to the receivers.\\
\subsection{Conclusion}
This section explains the logical structure of the system to implement the dual-feed receivers. Using the MPI-CH implementation under two parallel Raspberry Pi hosted applications, the system can achieve the goal of receiving dual feeds without implement expensive hardware which solves the problem by implementing parallel hardware bus between the CPU and two or more NIC cards. This section also explains how the three important parts works in the background: message queues, remote memory access implementation and fat-tree network topology. The next section will talk about how the third Raspberry Pi works as the master node connecting to a FPGA.\\
\section{Field Programmable Field Array(FPGA) Assisted Computation}
This section introduces the advanced hardware implementation on the $node_{3}$ discussed above. The Field Programmable Gate Array, FPGA is transitionally considered independent from the general computation architecture such as the CPUs, however, recent development in hardware pipeline technology enables the FPGA to communicate with the general architecture directly, and the content swap from FPGA to CPUs are much faster than the network supported data transmission.\\
FPGA is known for handling single threaded tasks which are hard to be split into sub-tasks, or the resources for parallel computing is much more costing than those for the single threading processes. FPGA is performs all the computation tasks via hardware directly, and they are programmable.\\
The project implements FPGA to support the \textit{merging} process after the cluster receiving the data feed. Discussion below shows the details of following technologies:
\begin{itemize}
	\item Sparta-6 integration with Raspberry Pi
	\item FPGA programming technology
	\item Communication between the CPU and FPGA: wishbone bus
\end{itemize}
\subsection{Sparta-6 integration with Raspberry Pi}
We put a Sparta-6 FPGA chip over $node_{3}$ which is an Raspberry Pi2 board. The firm ValentFX has created an integrated environment for the FPGA to run on the Raspberry Pi boards on Linux based systems. \\
First we need to understand the differences among programs running on three different platforms: CPU platform, micro controller platform and the FPGA platform. \\
\subsubsection{Programming on CPU platform}
Programs running on the CPU platforms are the easiest to understand: the programmers, in most cases do not need to worry the details of electric signals inside the CPU chips, and they only need to focus on the high-level logic such as giving an integer value to a parameter $a$ and add one to it for ten times. The programmers do not know how $a$ looks like in the CPU and how the logic 'add one to $a$ for ten times' works in sequence in the accumulators, data swapping in first level caches and system clock triggers. \\
\subsubsection{The MCU platform}
\begin{figure}
\centering
\includegraphics[width=0.6\linewidth]{picture/mcu}
\caption{Arduino MCU}
\label{fig:mcu}
\end{figure}
The micro controller platforms are similar to the FPGA ones, and in modern times, the MCU producers will release tools called MicroController Abstract for the programmers to \textit{translate} the high-level programming language, normally C, into the hardware languages that micro controllers can understand. Programmers know the architecture of the MCU their programs run on, and uses modern programming languages to manipulate the signals and clock triggers. Both the input and output of the MCU can be \textit{translated} by the MicroController Abstract layer into the high level programming format. Take the Arduino MCU platform for example: the MCU can produce six modulated electric pulses on its six pins showed in picture \ref{fig:mcu}.\\
The MCU works in this way: in the setup phase, the program set serial frequency to be 9600 bit per second as the data frequency of the pins, then the MCU is read for sending and receiving signals for each available pin, and in this example we read the default value of pin $PD3$ by using the method analogRead(). This function in the hardware abstract layer will be translated to a pulse which goes into the circus, and gain the module of ping $PD3$ signal within a system circle, and goes back to the hardware abstract layer. They layer will again translate the signals into a high level programming format, for example, if the black box attaches the $PD3$ pin to a high voltage source of 5 Volt, the abstract layer will translate the module 5 Volt into an integer of one.\\
The expression of different signals can be fixed by the manufacture or defined by the programmers, however, the key idea is that the black box which controls the behaviours of pins must be known to the system designers, and everything inside the black box is fixed and unchangeable, especially the modular logics within.Therefore, the design process of the MCU platform programs are:
\begin{itemize}
	\item[1.] Choose and understand the hardware logics of the MCU board
	\item[2.] Design the software which uses the functions provided by the hardware
\end{itemize}
A good example are the MCU chips in the heat sensors, which receives the signals of sensors it attaches to and the outputs are distance from the source to sensor, and a pair of coordinates indicating the relative positions of source. Programmers can make use of the outputs to design their programs, varying from an intelligent flushing toilet controller to navigating program in heat guided missiles.\\ 
\subsubsection{The FPGA platform}
The FPGA and microcontroller platform is quite similar, as both needs to know the detailed hardware implementations, and a hardware abstract layer to translate the programming languages and runtime signals.\\
The difference between the two is FPGA platforms do not have a \textit{fixed} module logics showed in the picture \ref{fig:mcu}, and the designing process of FPGA platform has two phases:
\begin{itemize}
	\item[1.] Design the module logics
	\item[2.] Design the software which uses the module logics in the FPGA
\end{itemize}
This process looks complex in the first place, because the module logic in FPGA chip is designed by the programmer, so the software needs to contain a full implementation of hardware abstract layer to perform the bi-directional translation. Programming on FPGA was difficult in the early times, however, the modern design of FPGA boards absorbs the advantages in MCU and FPGA, and creates an structure illustrated in the following picture:
\begin{figure}[H]
\centering
\includegraphics[width=0.5\linewidth]{picture/fpgadesign}
\caption{Modern FPGA design}
\label{fig:fpgadesign}
\end{figure}
The programmer nowadays do not need to worry about the hardware interfaces of FPGA chip, as the manufacturers have implemented an MCU over the FPGA, and setup the interfaces via MCU instead of the FPGA directly. As discussed above, the MCU has fixed interfaces, and the MCU on the FPGA board provides fixed general interfaces for applications to invoke, and fetch the data from FPGA. The advantage of using this middle ware structure is that the programmers do not need to care about the hardware implementation of the FPGA itself, and manufacturers can provide general drivers for the MCU over specific operating systems, with which the FPGA computation resources can be abstracted into an function in high level programming languages.\\
\subsubsection{Spartan-6 with Raspberry Pi2}
The Spartan-6 FPGA over Raspberry Pi, the Logi Pi follows similar way, and it does not use the MCU but rather a technology called to create a seamless communication between the CPU and FPGA. It uses Spartan-6 FPGA chip the hardware interfaces to connect to the Raspberry Pi board, supporting Linux based system, and programs can be burnt into the FPGA directly via general tools.\\
The advantage of using Wishbone will be discussed later, and for introduction, the Wishbone abstracts the MCU discussed above for running programs in FPGA and communicate with CPU. It is an abstract layer which now support Python and C programming language.\\

\subsection{FPGA hardware programming}
As discussed above, the FPGA programming involves two aspects, the software part which communicates with the FPGA, and the FPGA module logic which simulates the function of a MCU. This section will discuss the latter part, and show how this project works with the FPGA hardware design.\\
Program the hardware logic module in FPGA forms the following steps:
\begin{itemize}
	\item Understand the board structure: input and output signals
	\item Design the hardware logic inside the FPGA chip
	\item Programming the logic into FPGA and test
\end{itemize}
\subsubsection{Board structure}
The first step is to understand the board design, including understand the hardware resources we have on the board. For example, we need an accumulator to perform a for-loop, however, we need to know which pin refers to the clock on board, to trigger every step of the loop. In the Logi Pi v1.5 board, the clock can be triggered by an oscillator with standard 50MHz signal source, and connects to the $P85$ pin. \\
Usually, the pins of the FPGA chip are bounded into banks, so the electric schema of FPGA chips will show the pin structure as a form of banks. The FPGA manufacturers do not know the components outside the chip itself, however, only the creator of FPGA board can know how the board structure looks like. The design of FPGA application functions needs a file describes it, to avoid checking schematics diagrams every time, including the number, name and function of each pin on the FPGA chip, therefore we can use the hardware description language(HDL) to implement our functions.\\
\subsubsection{Design the hardware logic}
The second step is to understand the function need to implement: how the signals are imported into the FPGA, and how the output is consumed. The signals are variables that are created in the applications running on the CPU platform, and we have the Wishbone mechanism to slurp the signals from the memory in CPU to the Synchronous Dynamic Random Accessing Memory on the FPGA board, and push them into the designed pin. \\
Design of the hardware logic in FPGA follows the idea of modular programming, and the designer needs to split the hardware design into basic modules according to the functions FPGA chip provided. Each module is consisted of three parts: the input, which can be a system signal like clock, or the output from another module; the basic logic which contains the hardware logic of how a signal is processed by the basic components, for example the pulse-width modulation logic to carry a constant signal with a periodic clock using the following logic:
\begin{itemize}
	\item[1.] Defining the input signals with fixed bits
	\item[2.] Defining the parameter which receives the system clock signal, we can treat this parameter as an instance of class 'clock', which will change the value of itself according to time, and have methods to trigger an event which can be captured by system
	\item[3.] Defining the output, including the bits it contains and pin number it attaches to, the mapping of pin number and actual function, for example, which pin attaches to the LED, can be found on the schematics diagram
	\item[4.] Making a loop triggered by the clock, and define the behaviours of output inside the body
\end{itemize}	
The example procedure shows how to carry an eight-bit long constant signal with the provided system clock, and can be presented in the picture \ref{fig:pwm}. This diagram is the basic component of the hardware design, which we introduces to illustrate how the basic components discussed above are connected together as an overall hardware logic, to perform certain algorithm.\\
The benefit of using the modular design is that the modules are reusable, and the basic logic components inside FPGA is barely changed, so we can always reuse some classic solution even the code is decades ago.\\
The Logi Pi FPGA chip supports most of the modular design in the VHDL library, and the manufacturer developed some customized interfaces mainly for chip-to-chip communication especially for the ARM based platforms via PCI ports. These interfaces are the basic parts of this project, called Wishbone and will be discussed later.\\  
\begin{figure}[H]
\centering
\includegraphics[width=0.5\linewidth]{picture/pwm}
\caption{PWM modular logic}
\label{fig:pwm}
\end{figure}
\subsubsection{Burning the design into the FPGA chip}
The FPGA chips are reprogrammable, and the programmer needs to make the chip aware the design using a hardware-based descriptive language to make the FPGA chips know how to form the proper hardware structures. There are several different hardware description languages, and them main stream are the Verilog and VHDL. This project will not introduce or compare these two HDLs, as both are well-known and widely implemented over both academic and commercial fields. Most of the Logi Pi modular interfaces are designed using VHDL, and the project needs to reuse some of them, we choose VHDL as our programming language.\\
The hardware description language abstracts the hardware details, to improve the universality of the designs. Therefore the module design of input/output signals need a mapping procedure to find the correct hardware resources they combine to. This mapping is based on the descriptive file of the board mentioned above.\\
Design of the hardware logics always include reusable modules, and it is less likely that designers need to focus on the implementation of a simple module like PWM modules, and a library of implementation details is required when burning the design into the actual FPGA chips.\\
We have already known the board structure including input/output signal environments and the FPGA chip functions, we need to combine these information together to make the FPGA work as we designed:
\begin{itemize}
	\item Write and test the VHDL described design off board
	\item Introduce the hardware description file with UCF extension to link the modules to hardware interfaces
	\item Introduce the library to complete the detail design
\end{itemize}
The process now can be performed by the integrated development environment provided by Xilinx ISE tools. However, as the Spartan-6 is an old product and the ISE design suite is no longer supported by the Xilinx, there will be some difficulty in compiling the code, and will be explained in the system implementation chapter.\\
\begin{figure}[H]
\centering
\includegraphics[width=0.7\linewidth]{picture/vhdlstruct}
\caption{VHDL file structure}
\label{fig:vhdlstruct}
\end{figure}
\subsection{Wishbone: interface for FPGA-to-PC communication}
We had a brief introduction in how the FPGA works in the above sections, however, the project requires communication between an CPU hosted application and a FPGA hosted one. An commonly used solution is the serial communication architecture, and the communication is via the universal asynchronous receiver/transmitter interfaces\cite{michael1992universal}. The architecture is in picture \ref{fig:uart}.However, the UART technology is an overkill in this project, as the Logi Pi FPGA board is connected to the Raspberry Pi via the GPIO ports, and we can use the serial peripheral interface bus to implement the communication instead.\\
\begin{figure}[H]
	\centering
	\includegraphics[width=0.4\linewidth]{picture/uart}
	\caption{UART architecture}
	\label{fig:uart}
\end{figure}

\subsubsection{Serial Peripheral Interface on Raspberry Pi}
SPI solves the problem of visibility of FPGA modular details on the application side. Without SPI, the applications need to know the exact hardware layers between the FPGA pins and the GPIO ports it connects to, and all the read/write performances need to call a chain of hardware data drivers to make the communication work. However, the SPI isolates the banks by tagging the FPGA pins with addresses, and ideally, the applications on CPU side only need to know the virtual address it connects to, and the SPI layer will deal with rest of the details.\\ 
The structure of SPI is quite simple, which is consisted of a master and a slave. The master is embedded on the Raspberry Pi board forming two pairs of communication channel, a slave select channel and a clock:
\begin{itemize}
	\item Master Output Slave Input: data is the output of the master SPI and the slave receives the data
	\item Master Input Slave Output: master SPI receives, or reads the data from slave, however, the slave normally cannot invoke this channel
	\item Slave Select: the master SPI can select different slaves if there were multiple implementation of slaves in the FPGA
	\item Serial Clock: the master can output an clock signal instead of the build-in clock signal on the board
\end{itemize}
The Raspberry Pi disables the SPI driver by default\cite{rasppispi}, however, it is easy to unlock it by using the tool \textit{RASPI-CONFIG} provided by Alex Bradbury to unblock it\cite{rasppiconfig}. The communication of SPI showed in schematic diagrams \ref{fig:spi}. The SPI slave port connects to both the Spartan 6 FPGA chip and the SDRAM chip on the Logi Pi board, and we can use the SDRAM to store big data.\\
\begin{figure}[H]
\centering
\includegraphics[width=0.5\linewidth]{picture/spi}
\caption{SPI schematic diagram}
\label{fig:spi}
\end{figure}
\subsubsection{Implementation of Wishbone on Logi Pi}
The SPI solve the problem of data driver using GPIO ports communication, however, the implementation over pure SPI is still too complicate to achieve. The application on the CPU side still needs to know the implementation details of the FPGA side, for example, the choice of SPI slaves is triggered by the application, however, different FPGA chip may have varieties of slave memory access address according to the compiling environment for the VHDL. Wishbone technology provides an alternative solution that wraps up the SPI interconnection details into high-level programming language libraries, and combined with address mapping tool provided, the applications on CPU side can have a relatively \textit{fixed} call addresses.\\
The ValentFX has implemented Wishbone libraries in C and Python\cite{wishbone} with the above design, sacrificing some flexibility and providing much convince as designers do not need to reimplement any communication interfaces any more. The implementation of the libraries abstracts all the low level drivers and kernel details between the Raspberry Pi hardware and Logi Pi hardware, showed in picture \ref{fig:LOGIstack-software-drivers}).\\
\begin{figure}[H]
	\centering
	\includegraphics[width=0.2\linewidth]{"picture/LOGI stack - software - drivers"}
	\caption{Wishbone register}\cite{logipimanual}
	\label{fig:LOGIstack-software-drivers}
\end{figure}
The basic component of Wishbone is the register, which has a relatively \textit{fixed} address usually starts with $0x0$ with an offset $0x1$. It is the basic interface for all the wishbone logics and has 16-bit input/output\cite{logiwishbone}.\\
As mentioned above, the input and output processes are abstracted by the Wishbone wrapper. Take Python for example, reading and writing with register $0x0$ can be performed using the two functions provided by the library \textit{logiRead} and \textit{logiWrite}:
\begin{table}[H]
	\centering
	\caption{Wishbone functions}	\label{my-label}
	\begin{tabular}{@{}lll@{}}
		\toprule
		Parameter                                                    & Function                                                     & Notes                                                                                                                                                                                       \\ \midrule
		address                                                      & \begin{tabular}[c]{@{}l@{}}logiWrite\\ logiRead\end{tabular} & the read/write destination address in hex format                                                                                                                                            \\
		\begin{tabular}[c]{@{}l@{}}high-byte\\ low-byte\end{tabular} & logiWrite                                                    & \begin{tabular}[c]{@{}l@{}}the write content in hex format and the string will\\  be converted to ASCII. It has a pair of parameters \\ because the registers are 2 bytes long\end{tabular} \\
		length                                                       & logiRead                                                     & define the byte it reads, 2 means read the entire 16 bits                                                                                                                                   \\ \bottomrule
	\end{tabular}
\end{table}
This project does not have any peripherals and the hardware algorithms are implemented via the basic input/output processes. So the application on the MPI side needs to communicate with the registers via Wishbone, and the algorithms are performed after the data is sent to the registers. After receiving the data from Wishbone bus, we can implement the hardware algorithms afterwards.\\
The naming rules for the IO process in Wishbone follows:
\begin{itemize}
	\item Each pin is abstracted into an address in the SDRAM, and the FPGA performs IO sessions with SDRAM instead of GPIOs directly
	\item The SDRAM session with SPI is two sided, and the SPI can perform both read and write to a pin by calling the same address
	\item The address-mapping is hardware irrelevant, therefore different SPI or Operating System implementation will not affect the CPU side applications
\end{itemize}
\subsubsection{Parallel process and FPGA status}
The FPGA processes are running parallel with the applications on CPU side, and the application design should take this into consideration. The processing speed of FPGA is much faster than the normal CPU hosted applications, and the clock frequency is different between the cluster. The Wishbone mechanism synchronise the clock via transmission layer, however, the application inside the FPGA still needs to wait for the read signal from the application.\\
Another problem is that the status of FPGA is controlled by the electric signals only, and the interfaces are fixed. Therefore if the application on the CPU does not read from the FPGA first and change the status of FPGA, the processing result in FPGA modular will soon lost before the CPU applications can react.\\
These two features requires the design of the system to be aware of the execution sequences especially in the FPGA communication.\\
\subsection{Conclusion}
This section introduces how the FPGA will work with the receiver applications on the MPI-CH structure, and focuses on how to design and implement a hardware algorithm in the FPGA, from design to actual coding, and the important technology which enables the hardware to seamlessly communicate to the CPU hosted applications using Wishbone. The FPGA accelerated computation is widely used in the commercial solution to similar problems, especially in the market data feed handlers, however, this project shows the potential of using an open-source FPGA board to achieve the same goal over an cluster hardware environment.\\ 
\section{Conclusions for the Chapter}
In this chapter, we introduced the core technologies we implemented in the project, and focuses on the Message Passing Interface; the Remote Direct Memory Access method and the reason of implementing Fat-tree network with explanation of its benefits; the Hardware design and programming technology of FPGA and focuses on how the hardware can communicate with the software running in the modern CPU supported system.\\
We draw a blueprint of a hardware platform which is linked by Ethernet, and the software runs on it is supported by MPI. The system is scalable and efficient because we implement the Fat-tree network with adjustment of interconnection topology and bandwidth.\\
The following chapter will discuss the system design and the real implementation of the experiment, and how the features are brought to life.
\chapter{Project Design and Implementation}
This part of the thesis explains the main objective of the experiments, design of the experimenting system, and the implementations of technologies discussed above. 
This part is consisted of three parts:
% Please add the following required packages to your document preamble:
% \usepackage{booktabs}
\begin{table}[H]
	\centering
	\caption{Content structure of system design and implementation}
	\label{my-label}
	\begin{tabular}{@{}l|l@{}}
		\toprule
		Name                                                                                    & Content                                                                                                                                                                     \\ \midrule
		\begin{tabular}[c]{@{}l@{}}Experiment Objectives and\\System Requirements    \end{tabular}                                                                 & \begin{tabular}[c]{@{}l@{}}Showing system requirements based on the hardware\\ achieved, and simulation goals\end{tabular}                                                  \\ \hline
		\begin{tabular}[c]{@{}l@{}}Hardware Design and\\ Implementations\end{tabular}           & \begin{tabular}[c]{@{}l@{}}Showing the hardware design of the system, including \\ connectivity implementations and power supply \\ implementation\end{tabular}             \\ \hline
		\begin{tabular}[c]{@{}l@{}}Software Design and\\ Implementations\end{tabular}           & \begin{tabular}[c]{@{}l@{}}Showing the design of different sender-receiver pair, \\ with and without MPI layer and FPGA assisted ones\end{tabular}                          \\ \hline
		\begin{tabular}[c]{@{}l@{}}Hardware Algorithm\\ Design and Implementations\end{tabular} & \begin{tabular}[c]{@{}l@{}}Showing the design of the hardware implementation \\ along with the software interfaces of the application \\ on the Spartan-6 FPGA\end{tabular} \\ \bottomrule
	\end{tabular}
\end{table}
\newpage
\section{Experiment Objectives}
\subsection{Overall Design}
The project designed an experimental system over Raspberry Pi 2 Model B hardware, that supports 100Mbps Ethernet dual UDP datagram feeds as data source. The system should be flexible in error tolerance such as package loss, and scalable in implementing additional heterogeneous hardware. The network can be separated into two parts: the communication layer that supports low-latency network, and monitor system embedded in the SoC that implements additional features such as error tolerance. \\ 
The System over Chips(SoC) system are interconnected as a homogeneous processing cluster. The task/thread management are performed by interconnected CPUs over multi-processor interfaces, memory solution for bandwidth requirement of high frequency data is RDMA protocols implemented over open sourced hardware memories. The FPGA attached in the master node is in charge of decoding, decompression and filtering for dual feeds.\\
\begin{figure}[H]
	\centering\includegraphics[width=1.0\linewidth]{picture/System_Design.jpg}
	\caption{System blueprint}
	\label{fig:system_design}
\end{figure}
This solution implements and improves the solution provided by the commercial firms in 10Gb Ethernet Switches, which introduces high performance computation chips to resolve the pressure in the network interface buffer. The improvement is in the layer of the high performance computing: in this solution, the HPC layer is not linked directly to the NIC in the board, but is interconnected by the Ethernet Switch, which is much cheaper that 10Gb Ethernet switches. It sacrifices bandwidth in inner system to achieve similar efficiency in data processing by give spaces to heterogeneous scalable computational platforms, consisted by open sourced hardware like Raspberry Pi, and the fat-tree topology interconnection enables the cluster to expand without much cost.\\
The data flow of the system is demonstrated in picture \ref{fig:Systemdataflowdiagram-Page1}.
\begin{figure}[H]
\centering
\includegraphics[width=1.0\linewidth]{"picture/System data flow diagram - Page 1"}
\caption{System data flow diagram}
\label{fig:Systemdataflowdiagram-Page1}
\end{figure}
This data flow diagram is illustrated in BPMN 2.0 standard, and the legends will be attached in the appendix.\\
The software system is consisted of three parts:\textit{Green} area: the sender, which sends high frequency dual data feeds;\textit{Yellow} area: the data handler. This one has two implementations, receiving A-B data feeds discussed in the first chapter; \textit{Purple} area: the data controller. This one aggregates the output from the handlers, and performs verification of received data. The channels among the system is maintained by two different architecture: the communication between the sender and handler, and controller to sender are maintained by UDP sessions. The intercommunication between the handlers and controller is embedded on the MPI-CH2 architecture.\\
\subsection{Expectations}
The system can be implemented as the interconnection switch network for cloud computing clusters and high performance computer clusters. The design introduces the infrastructural solution for low-latency market data feed, utilizing data transmission efficiency with limited bandwidth, and the fat-tree topology network provides features in system monitoring and scalability in supporting massive computation.\\  
This thesis aims at providing a possible solution to this problem, which is implementing the dual data feed system over the open sourced hardware, and using Message Passing Interface technology to make the system scalable to achieve similar performance as the commercial ones with lower cost in purchasing and energy cost.\\
\subsection{Main Objective}
The main object for the project is to implement an experiment over a combined hardware-software environment with the features showing in table \ref{table:features}. 
\begin{table}[H]
	\centering
	\caption{System Features}
	\label{table:features}
	\begin{tabular}{@{}l|l@{}}
		\toprule
		Features                                                                           & Description                                                                                                                                                                                                                                               \\ \midrule
		\begin{tabular}[c]{@{}l@{}}A-B UDP data feed\\ handler\end{tabular}                    & \begin{tabular}[c]{@{}l@{}}The system needs to support dual data feed(A-B data feed)\\ to simulate the senario of ultra-low latency market data \\ transmission\end{tabular}                                                                              \\ \hline
		\begin{tabular}[c]{@{}l@{}}FPGA accelerated\\ UDP package processsing\end{tabular} & \begin{tabular}[c]{@{}l@{}}The decompression and filtering of dual UDP packages in \\ FPGA are proved to be more efficient than pure CPU \\ solutions, and can be expand to more general situations\\ in normal high frequency data networks\end{tabular} \\ \hline
		\begin{tabular}[c]{@{}l@{}}ARM based SoC \\ interconnection\end{tabular}           & \begin{tabular}[c]{@{}l@{}}Interconnection among ARM based SoC network makes it \\ possible to implement scalable buffer capacity and bandwidth\end{tabular}                                                                                              \\ \hline
		\begin{tabular}[c]{@{}l@{}}Fat tree \\ interconnect \\ network\end{tabular}        & \begin{tabular}[c]{@{}l@{}}Fat tree topology is proved to be efficient in \\ implementing interconnection network for massive parallel \\ computation, and flexible in scalable computational nodes.\end{tabular}                                         \\ \bottomrule
	\end{tabular}
\end{table}
The system is consisted of two parts:\\
\begin{itemize}
	\item The hardware part: creating physical connection among ARM based SoCs and FPGA accelerated data processor, and connect to the data feed via Ethernet cables;
	\item The software part: creating logical fat-tree topology network, that buffering dual feeds in the slave nodes and processing UDP packages via FPGA accelerated master node and other parallel user cores; 
\end{itemize}
\section{System Requirements}
This project wants to achieve the overall following goals:
\begin{itemize}
	\item Inputs of the system, including the dual-feed hardware structure, and UDP session simulation
	\item FPGA accelerated package filtering: the logic and hardware implementation
	\item Monitoring and printing the send-receive results of the communication
\end{itemize}
The project does not implement the hardware equipments of 10Gb Ethernet because of the complexity in constructing the hardware over Raspberry Pi and the server side. In order to simulate the environment using RJ-45 cables, we implement a simulated software environment which manually configure the package loss, sending/receiving window and other scenarios.\\

\subsection{System Input: Dual feeds and UDP Session Simulation}
This session explains the requirements in system input, including how to form the hardware input of dual feeds from the simulated server side, and the UDP session simulation, especially how to choose the package loss ratio.\\
\subsubsection{Dual-feed receivers hardware}
The main goal of this section is to assembling a hardware using Raspberry Pi boards that can receive two network signals from two RJ-45 cables. The bandwidth of both feeds are up to 500 Mb, however, this includes the bandwidth of incoming data and interconnection communication.\\
In order to measure the experiment results in an easy way, we simulate the server side with only one source, and the offset of package is manually set up in the sender application. Signal of the sender is split by an Ethernet Switch.\\

\subsubsection{Sender-receiver pair requirements}
The sender application should be able to differ the receiver with manual offset, and the receiver needs to be able to implement the following features respectively:
\begin{itemize}
	\item Receive the data feeds and perform error-check without communication between the two feeds
	\item Receive the data feeds and perform error-check after merging the feeds, this requires communication between the nodes, and this should be separated into two scenarios: with and without MPI communication channel support
	\item Receive the data feeds and introduces the FPGA algorithm to perform the error check
\end{itemize}

\subsubsection{UDP simulator}
This section simulates scenarios of UDP transmission in low-latency network. According to the Cisco user manual of the commercial high-frequency market data handler, the metrics of the bandwidth and message rate follows\cite{ibm2011highfrequency}:
\begin{table}[H]
	\centering
	\caption{Low-latency market data handler metrics}
	\label{tab:lowlmdm}
	\begin{tabular}{@{}lll@{}}
		\toprule
		Name                      & RTT & \begin{tabular}[c]{@{}l@{}}Average Time Cost Per Hop\\ {[}$usec${]}\end{tabular} \\ \midrule
		No watchlist overlap      & 1   & 6                                                                                \\
		Custome watchlist overlap & 1   & 7                                                                                \\ \bottomrule
	\end{tabular}
\end{table}
It is clear that the 10Gb Ethernet bandwidth and data speed is higher than the RJ-45 maximum metrics. The experiment, however, introduces the possibility in packet loss of UDP datagrams, and scaling down the size of required bandwidth and message rate with corresponding package loss ratio, to prove that if we implement similar solution over the 10Gb Ethernet using optic fibre with the same Raspberry Pi structure, we can still achieve the results close to the experiments.\\
The calculation of packet loss ratio of UDP datagrams $P_{UDP}$ in single feed follows with the listing definitions\cite{moon1998correlation}:
\begin{itemize}
	\item $N_{send}$ is the packages sent from the sender side
	\item $N_{rec}$ is the package received in the receiver side
	\item $Msg_{i}$ defines the status of a package $i$:
\end{itemize}
\begin{equation}
	Msg_{i}=\begin{cases}1 & packet\ i\ is\ received
	\\
	0 & packet\ i\ is\ lost
	\end{cases}
\end{equation}
\begin{itemize}
	\item $delay_{i}$ is the delay of packet $i$
\end{itemize}
Therefore, it is easy to understand that the average package loss ratio $\overline{l}$ is $\frac{\sum_{N}^{0}Msg_{i}}{N}$, however, in this project the scenario is a bit different, because in high-frequency network, the receiver implements a time-out more restricted to normal network threshold recommended\cite{eggert2009tcp}. The real package loss ratio in this project should be:
\begin{equation}
	\overline{{l}'}=\frac{\sum_{0}^{N}Msg_{i}-\sum_{0}^{N} H(delay_{i}-t_{threshold})}{N}
\end{equation}
\begin{equation}
	H(x)=\begin{cases}
	1 & x>0
	\\
	0 & x<0
	\end{cases}
\end{equation}
So the accuracy of the experiment result now lies in the estimation of normal package lost rate and package overtime rate corresponding to the chosen bandwidth, and simulate the environment of 10Gb Ethernet under the simulated experiment environment. Therefore, we can estimate the real UDP package reliability in low-latency network with the simulated package lost $Msg_{i}$ and overtime packages whose $delay_{i}$ is larger than the fixed threshold in the receiver.\\
\subsubsection{Value of the packet loss rate}
According to a research in the UDP multi-cast network, the package loss rate in multi-cast network of UDP protocol can be expressed in the following way\cite{caceres1999multicast}:
Given the definitions listed below:
\begin{itemize}
	\item $V$ is the overall network using the UDP multicast
	\item $\alpha$ is the given node set status when the node lost packet, and $\alpha = \alpha_{k},k\subseteq V$
	\item $\chi{i}$ is the $i$th independent observation result of a node, where $\chi > 0$ means it receives the package, and $\chi =0$ means it does not receive it
	\item $\Omega$ is the subset of $V$ whose nodes loses packages
\end{itemize} 
We have the maximum likelihood estimation of packet loss in this network:
\begin{equation}
	p(\chi{0}...\chi{n}|\alpha)=\prod_{0}^{N}p(\chi{i}|\alpha)
\end{equation}
It will be convenient to calculate the overall package loss if we manually set the observation result of single node $\chi{i}$ to a step equation, therefore $\chi{i}=\begin{cases}1 & \chi = 0 \\ 0 & \chi > 0\end{cases}$. In order to make the calculation easier, we take the logarithm of likelihood forming a log-likelihood estimation:
\begin{equation}
	\eta(p)=log\prod_{0}^{N}p(\chi{i}|\alpha)=\sum^{\Omega}log(p(\chi{i}|\alpha))
\end{equation} 
Therefore, we need to find out the measurement method of independent observation of package loss in each node $p(\chi{i}|\alpha)$, however, this value only relates to two factors: the datagram size and predicted send time. These two factors are known to us from the above discussion measured from Cisco, and we need to find out an empirical formula to derive the value of it.\\
Considering a typical UDP send-receive process, we divide the possible scenarios into two:
\begin{itemize}
	\item No package loss
	\item Package loss occurs
\end{itemize}
Given an equation of required bandwidth $B_{req}=\frac{S_{pack}}{T_{trans}}$, where $S_{pack}$ is the total size of the packages and $T_{trans}$ is the estimated time transmitting the whole package. We introduces another time variable specifically for the propagation time $T_{prop}$ In the above two scenarios, we have:
\begin{itemize}
	\item No data loss: $T_{totoal}=T_{prop}+\frac{S_{total}}{B_{send}}$
	\item With data loss: ${T}'_{total}=T_{total}+T_{resend}$, and we have $T_{resend}$ equals to the time cost in resending UDP datagram
\end{itemize}
\begin{equation}
T_{resend}=\frac{S_total}{S_{datagram}}\times p(\chi|\alpha) \times T_{prop} + \frac{p(\chi | \alpha) \times S_{total}}{B_{send}}
\end{equation}
Assuming that UDP datagram lost will happen in most cases, therefore the second scenario can be viewed as the average value of $T_{total}.$From table \ref{tab:lowlmdm} we have:
\begin{itemize}
	\item Minimum requirement: $S_{send}=45B$, ${T}'_{total}=6usec$, $T_{prop}=1usec$
	\item Maximum requirement: $S_(send)=45B$, ${T}'_{total}=7usec$, $T_{prop}=1usec$
\end{itemize} 
The time consumption for resending the package is relatively small, and $\frac{p(\chi | \alpha) \times S_{total}}{B_{send}}$ can be omitted. We assume that the processing speed is similar in both commercial environment in 10Gb Ethernet and the experiment environment, and the fixed number is $t_{trans}$. The data handlers uses 2 Intel® Xeon® (X5570) 4-core 2.93GHz processors\cite{ibm2011highfrequency}, and the estimated processing time is $t_{trans} \approx 5.8usec$, and ${t}'_{trans} \approx 6.82usec$ therefore we have estimation: 
\begin{equation}
	p(\chi | \alpha) \subseteq [2.639\%, 3.448\%] 
\end{equation}
\subsubsection{Choice of two receiving strategy}
There are two different receiving strategies according to the distribution of filtering UDP packages. Filtering is the process for checking each piece of data feed datagram $i$: 
\begin{itemize}
	\item Has the receiver received datagram $i-1$
	\item Is the content in datagram $i$ expected
\end{itemize}
The first strategy is to implement filtering in each data feed receiver:
\begin{figure}[H]
\centering
\includegraphics[width=1.0\linewidth]{"picture/Filtering in handler - Page 1"}
\caption{Filtering in handler data flow diagram}
\label{fig:Filteringinhandler-Page1}
\end{figure}
For each data receiver A and B, we need to wait each to finish, or time out, then the results will be retransmitted to the controller. The filtering takes time, and the status query between the receivers will affect the efficiency of the system. Another potential harm is that one or more data feeds lost connectivity and the alive session needs to wait until the dead ones expired then to retransmit. This will significantly increase the latency in the controller, as the open sourced hardware is not reliable as the commercial ones, and the system needs to be able to tolerate this situation.\\
We implement the second choice: this one is to perform filtering both in the controller, and the receiver only in charge of individual feed, and the data will be retransmitted to the controller right after the session is over. This solution solves the problem in status queries among receivers, as they do not need to know the status of others, and dead connections will not affect the overall efficiency, as the controller hosts the keep-alive sessions among nodes parallel to the receiving sessions. The worse expired receiver is the one dies in the last package, and the controller only needs to wait for $T_{alive}+T_{offset}$, where $T_{alive}$ is the maximum keep-alive time for each node, and $T_{offset}$ is the manual offset between different data sources.\\
\subsubsection{Conclusion}
We derive the data loss ratio in high-frequency network for UDP datagrams should be  [2.639\%, 3.448\%], with the following assumption:
\begin{itemize}
	\item The data loss rate in UDP datagram is similar in both experiment environment and real scenarios
	\item The ratio of data processing time and propagation time can be considered fixed if the size of datagrams are similar
\end{itemize}
We also makes decision of how the data feed receiver works, especially how the filtering process is distributed. We decided to distribute the filtering in the controller together, and the receiver only needs to retransmit the received datagrams regardless the status of other receivers.
\newpage
\section{Hardware design}
The table of materials used in the project is shown in table \ref{bom}. However, we do not have 
\begin{table}[H]
	\centering
	\caption{Materials used in the experiment}
	\label{bom}
	\begin{tabular}{@{}|l|l|l|l|@{}}
		\toprule
		Name                  & Amount & Manufacture    & Metrics                                                                                                                    \\ \midrule
		Raspberry Pi2         & 3      & Raspberry Org. & \begin{tabular}[c]{@{}l@{}}1x Boardcom ARM CPU, \\ 1x Ethernet Port, \\ 1x SD card port\\ 5.1V 1A input power\end{tabular} \\ \hline
		Kingston MicroSD card & 3      & Kingston       & 8GB Micro SD card                                                                                                          \\ \hline
		Logi Pi FPGA board    & 1      & ValentFX       & \begin{tabular}[c]{@{}l@{}}1x SPARTAN-6 FPGA chip, \\ 5V 1A input\end{tabular}                                             \\
		Ethernet Switch          & 1      &                & \begin{tabular}[c]{@{}l@{}}1x S1786 central chip, \\ 8x Ethernet ports\end{tabular}                                        \\ \hline
		Ethernet cable        & 4      &                & RJ45 copper cable                                                                                                          \\ \hline
		Power source          & 1      & iClever        & \begin{tabular}[c]{@{}l@{}}6x Powerports, \\ 100-240V 50-60Hz 1200mA input, \\ 5V(max) each output\end{tabular}            \\ \hline
		Power cable           & 4      &                & USB power cable                                                                                                            \\ \bottomrule
	\end{tabular}
\end{table}
\subsection{Interconnection Design and Implementation}
The Raspberry Pi only has one Ethernet Port, and data traffic among them can be achieved using two ways: the Ethernet cable and GPIO ports. The Ethernet cable way is easy to understand: Raspberry Pi boards forms a LAN and they can reach each other using Internet protocols. The GPIO ports is another ideal way for our project, because the data transmission ratio will not be limited by the interconnection architecture which overlays the incoming data feeds. This uses GPIO hardware communication structure which has been introduced in former chapter, and the applications in this solution exchange direct bit data via the hardware directly through the GPIO pins.\\
However, the implementation of UART interconnection has the following problem: the pin number of GPIO ports on Raspberry Pi boards is not big enough to process large quantity of data, and an solution for that is using an GPIO expansion board. We also need to implement an synchronizer between two GPIO ports because the system clock on different Raspberry Pi boards is slightly different, and will affect the efficiency in transmission.\\
\subsubsection{Interconnection design}
In order to perform the experiment with minimum cost, we decide to use the Ethernet connection solution among the Raspberry Pi boards shows in picture \ref{fig:hardwareDesign}.\\  
\begin{figure}[H]
\centering
\includegraphics[width=0.5\linewidth]{picture/hardwareDesign}
\caption{Interconnection hardware blueprint}
\label{fig:hardwareDesign}
\end{figure}
The bandwidth requirement has been discussed in the previous sections, and we can derive the overall bandwidth of each system:\\
\begin{itemize}
	\item Each port supports maximum 500Mbps
	\item RJ-45 cable only supports 100Mbps bandwidth\cite{kunz1999rj}
	\item The bandwidth inside switch can be considered as infinite, as it is inner-chip communication inside.
\end{itemize}
Therefore, the actual bandwidth limit in one-side communication is 100Mbps, however, the MPI-CH is a two-side communication among the nodes, and the server input also uses the same interconnection network. Therefore, the upper limit in actual one-side communication is less than 40Mbps.\\
\begin{figure}[H]
	\centering
	\includegraphics[width=0.5\linewidth]{picture/photo/SwitchConnection}
	\caption{Cabling on the switch board}
	\label{fig:SwitchConnection}
\end{figure}
\subsubsection{Interconnection implementation}
We have two sorts of Raspberry Pi boards, the slave nodes that perform as the receivers, and a master node which connects to an Logi Pi FPGA board. The network connection of these two kinds are similar:
\begin{figure}[H]
\centering
\includegraphics[width=0.9\linewidth]{picture/photo/rappCluster}
\caption{Ethernet connection in slave node and master node}
\label{fig:rappCluster}
\end{figure}
\subsection{Power Supply Design}
The power supply of the cluster is divided into two parts(the simulated server has independent power supply). The switch is supported directly from a power strip without backup, and the cluster is supported by an USB power hub with stabilizer. The design should be enough for experiment, however, as the Logi Pi FPGA board does not have any static memory module, the hardware content will be lost if the power supply is cut of for the master node, we should implement Unstoppable Power Supply units for large scale implementation.\\
The design of power supply is illustrated in the picture \ref{fig:powerDesign}. The FPGA can get power from both the GPIO pins in the master board, and the USB power hub with a standard A-B USB cable. The FPGA board is used for accelerating the processing speed and there is no meaning to keep it alive if the master node is off-line.\\
\begin{figure}[H]
\centering
\includegraphics[width=0.7\linewidth]{picture/powerDesign}
\caption{Power supply blueprint for the cluster}
\label{fig:powerDesign}
\end{figure}
\subsection{Conclusion}
We discussed the hardware design in this section, and set the bandwidth of the experiment environment to be $B_{exp} \approx 40Mbps$, and the detail description for the network structure implementation and power supply implementation prove that the experiment over Raspberry Pi-Logi Pi cluster can demonstrate the solution we provided for the high performance network receiving dual feeds.\\ 
\section{System Design and Implementation}
The blue print of the software system over the cluster is illustrated in picture \ref{fig:LogicDesign}, and the implementations are focused on the MPI supported Linux cluster over Raspberry Pi, including the installation and configuration of MPI-CH2 framework on Linux, configurations of Python support and FPGA development configuration.\\
The Raspberry Pi hardware, however, is different from general computing platforms in the following ways:
\begin{itemize}
	\item Only limited operating system images can be installed on Raspberry Pi, and most of them are deep customized for the ARM based platforms
	\item Hardware portals on the Raspberry Pi is different, including the SPI and i2c configurations
\end{itemize}
We encountered some legacy problems from both the cluster implementations and Windows system as the server. The solution for these problems will be discussed in this section.\\
\begin{figure}[H]
	\centering
	\includegraphics[width=0.5\linewidth]{picture/LogicDesign}
	\caption{System blueprint}
	\label{fig:LogicDesign}
\end{figure}
\subsection{Operating System and Internet Connection}
We chose the Raspbian from one of the official supported customized OS images by Raspberry Pi organization as our operating system. The Raspbian is customized from Debian to support the ARM CPUs, and other unique hardware drivers in Pi boards\cite{harrington2015learning}. Writing the system image into SD card and plug it into the Pi board, the system is well supported by the community that starting up is quite easy. However, it would occur some bugs in the system if we do not start up the cluster in the correct sequence.\\
\subsubsection{Solution to the IP address configuration}
 The Internet connection of the cluster is provided by the computer served as the sender. It shares the WIFI public network with the cluster using a wired cable linked from the computer to switch. After sharing the network, the IP of wired network adaptor in the computer will be fixed to 192.168.137.1 with net mask 255.255.255.0. We configure the Raspbians to manually get their fixed IP address on eth0 interface by changing the booting configurations, and the three nodes uses 192.168.137.101, 192.168.137.102, 192.168.137.103 in order to avoid conflict. This setting needs to be set in both booting configure and the image level file cmdline.txt.\\
 \subsubsection{Solution to Internet reconnection}
 We need Internet connection over the cluster to perform package installation and update, however, a bug in Windows 10 wireless network sharing policy can cause a lost in Internet support of the cluster. The bug happens when the OS is turned to sleep or when wireless signals are cut off for a while. The wired adapter will be turned into the primary adapter on the above scenarios, and some parts of the network sharing will not work even we manually reconnect the wireless network again.\\
 A solution to this is to perform ARP scan from the server over the clusters manually after reconnecting, and be sure that all the interfaces on the Raspberry Pi are set to manual configuration, because the wired adaptor may lose the setting of disabling DHCP, and some network interfaces might be activated by mistake.\\
\subsubsection{Interacting with the cluster: ssh terminal and VNC}
We choose two ways to interact with the cluster, ssh using Putty client and VNC to see the XWindows desktop. The latter one is implemented by the RealVNC viewer on the monitor side\cite{realvnc}, and tightvncserver on the cluster side\cite{xtightvncserver}.\\
\begin{figure}[H]
\centering
\includegraphics[width=0.9\linewidth]{picture/terminal}
\caption{SSH terminal and VNC XWindows for master node}
\label{fig:terminal}
\end{figure}
\subsection{MPI-CH2: Install and Configure}
MPI-CH2 needs to be compiled and installed from the source. The latest stable version for MPI-CH2 is 3.2. Installing of MPI-CH2 is time-consuming, and takes about three hours to compile, build and install the entire system without any error. Therefore we made one image with MPI-CH2 installed, and wrote that to all rest two nodes to save time.\\
This section will not discuss in detail about the installation steps and configuring system files, however, we can describe the overall process is to download source, build and compile, then add necessary middle wares, in the end the MPICH2 package provides example codes and will work if the installation is completed. The configuration of MPI-CH2 over Raspbian is a bit different from other x86 Linux images. The hostname of Raspbians are the same due to the default configuration, and changing the host name in the file /etc/hostname will affect the environment because some programs, including mpiexec and mpirun, will use etc/hosts as the indexing. The changing in IP address discussed above will make these applications unable to find the local host, as it is set to be 127.0.0.1 with the name 'localhost'.\\
\begin{figure}[H]
\centering
\includegraphics[width=0.9\linewidth]{picture/mpi-setup/setpsMPICH2}
\caption{Setting up the MPI-CH2 environment}
\label{fig:setpsMPICH2}
\end{figure}
\subsubsection{Adding Python support to MPI-CH2:MPI4PY}
Python has implemented MPI support in the language, in both Python 2 and Python 3\cite{mpi4py}. However, the library temporary supports only the OpenMPI, and we needs to manually add it to the MPI-CH architecture. Fortunately the MPI-CH has already prepared this scenario, and the MPI command environment supports the method \text{mpirun.openmpi} to execute applications using OpenMPI.\\
The Python MPI library, \text{MPI4PY}
\section{Software Design and Implementation}
The data flow blueprint for the project can be illustrated in the picture \cite{fig:LogicDesign}, and the implementation details can be very different according to the framework we are using. The software system can be split into three parts:
\begin{itemize}
	\item Data feed: simulates the high-frequency data, can be configured into different sending modes with related package size and error rate
	\item Dual-feed receiver: the receivers are running over the MPI-CH framework, however, it can have two different strategy in processing received data: the data verification mechanism happens in the master node, and the receivers can choose either to retransmit every single datagram right after receiving it, or retransmit the entire package after the transmission is time-out
	\item Master node: the master node needs to perform at least two functions: metrics collection and data verification
\end{itemize}
We demonstrate the solutions to these requirements in the following contents. Multiple implementations of data feed and master node are introduced in order to make comparison of combination in the experiment, and we make decision of the strategy in the dual-feed receiver for some experiments done on the cluster. 

\subsection{Design of the Data Feed}
The flow diagram of data feed is picture \ref{fig:sender-diagram}. It does not fully simulate the real A-B data feed over UDP protocols, however, because we send serials of integer data to the receivers in order to perform benchmark of the efficiency. 
\begin{figure}[H]
\centering
\includegraphics[width=0.5\linewidth]{picture/sender-diagram}
\caption{Data feed application blueprint}
\label{fig:sender-diagram}
\end{figure}
The main goal of the data source provider is to simulate the data loss, and we have two ways to implement it. One is to make a real manual data loss in some of the packages before sending, and leave the verification to differ the 'correct' data source and 'corrupted' data source. Another is to pre-set a bulk of independent sending tasks, and manually disable some of the tasks randomly.\\
The first option simulates the UDP datagram transmission directly, however, it is takes time to make comparison, as for each sending task, the receiver side needs to run a full verification. It also cannot perfectly simulate the situation with one of the source significantly delayed, because the UDP datagram can arrive in mixed sequence and the UDP receiver will automatically sort the data according to the header.\\
The second option uses failure rate of the independent replications to substitute the data loss rate in single transmission, with the definition of Bernoulli experiment\cite{anderberg2014cluster}. We choose this design as the data sender.\\
\subsubsection{Transforming data loss rate to session failure rate}
For a group of tasks with an amount of $N$, the failure rate $P=\frac{N_{fail}}{N}$, and for a single task, the failure rate $p_{fail}=P$, therefore, for $N$ times of experiments, the probability of failure happens $k$ times is: $C_{N}^{k}(1-p)^{n-k},\ k \subset [0,1,2...,N]$. According to the discussion in system requirements, we have the data loss ratio $p(\chi | \alpha) \subseteq [2.639\%, 3.448\%] $, and we transform this ratio into the success rate of $N$ times Bernoulli experiments by taking the $p(\chi | \alpha)$ into the equation of independent experiment probability.\\
The reason for this transforming is that the UDP protocol does not have an error check mechanism defined, and it is hard to benchmark efficiency in different data loss rate by comparing the success rate of transmission, as one data loss is a failure defined in the protocol, and the output of experiment $i$ for one time A to B UDP session with data loss is either success, or failure. Instead of measuring the success probability, we made a bundled independent experiments, and randomly fail some of them to simulate the data loss, and the receiver do not need to implement the error check mechanism as the lost data will be filtered before receiving, which saves a lot of time.\\
\begin{figure}[H]
\centering
\includegraphics[width=0.7\linewidth]{picture/udpsender}
\caption{UDP sender from data loss rate to success rate}
\label{fig:udpsender}
\end{figure}
\subsubsection{Data lost ratio randomize}
We shift the data lost in Ethernet transmissions into the manually set random data lost, which is implemented by software only. As discussed above, we need to find a reliable random number generator to simulate random data loss.\\
We believe that the data loss happens independently, as infinite cause will result in data loss during transmission. However, we have measured the normal data loss ratio of UDP datagram in 10Gb Ethernet infrastructure, and we have two possibility models to simulate: the Gamma Distribution and the Gaussian Distribution\cite{dubois2006possibility}.\\
This project, however, does not focus on the random number generation strategy, and we only need to have a brief analysis over the two different distributions. The Gaussian Distribution, or the Normal Distribution, describes how the independent experiment set $A$ will distribute as event $\alpha$ will most likely to occur in a certain possibility\cite{morrison1990multivariate}, but the Normal distribution will satisfy the conditions of Central Limit theorem only when there are finite variances to affect the event $\alpha$ \cite{rosenblatt1956central}, therefore the data loss in Ethernet transmission cannot meet the requirements of Gaussian Distribution because the causes are infinite, as discussed above.\\
The Gamma Distribution defines the possible time $t$ in a set of experiment $B$ if we want certain event $\beta$ to happen, and the reciprocal of $t$ can be viewed as the frequency of event $\beta$ happens in general\cite{stacy1962generalization}. The Gamma Distribution does not require the finite or infinite influential variables, and we have already got the possibility of the event \textit{data loss} happen in the discussion above, the Gamma Distribution is an ideal random simulation in this project.\\
The two variables in Gamma distribution should have the following value according to the discussion above:
\begin{itemize}
	\item Mean value $e$: $3.0435\%$
	\item Dispersion $\phi$: this value needs to be derived from the density function of Gamma distribution, and will be attached in the appendix. The value in the sender is $2$.
\end{itemize} 
\subsubsection{Kolmogorov-Smirnov test}
We introduced the Kolmogorov-Smirnov test(K-S test) to check if the data loss ratio fits the Gamma Distribution\cite{vlvcek2009daily}. K-S test implements the basic idea that if the distribution of  experiment results fits our pre-defined distribution pattern, then we can accept the fact that events happened in the experiment fits our defined distribution. The process of the K-S test is simple, and we introduce the following terms to show the statistics we need to collect.\\
\begin{itemize}
	\item[1.] $H_{0}$ is the assumption we make, that the data loss distributed in the experiment fits Gamma distribution
	\item[2.] $H_{1}$ is the reverse of the assumption, which assumes the results cannot fit the definition
	\item[3.]  $F_{0}(x)$ is the expression of Gamma Distribution, $F_{0}(x)=\frac{1}{\beta^{\alpha}\Gamma(\alpha)}x^{\alpha-1}e^{-\frac{x}{\beta}}$, $x>0$
	\item[4.] $F_{n}(X)$ is the frequency equation of event $k$ happens in the experiment result.
	\item[5.] $\alpha(0.05)$ significance level, and if $p-value$ is smaller than it, the distribution will not fit the assumption $H_{0}$
\end{itemize}
The calculation for the K-S test is to find the value pair $ D=\{max|F_{0}(x)-F_{n}(x)|, p-value\}$ \cite{rice1989analyzing}. Value of $D$ shows the approximation level between the experiment and ideal distribution, and the smaller it is, the higher it will be possible that the experiment fits the defined distribution pattern, and $p-value$ is defined to compare with the significance level, and if $p-value < \alpha(0.05)$, $H_{0}$ will not be real.\\
\subsubsection{Dual data sender implementation}
The dual data sender needs to implement the following features:
\begin{itemize}
	\item Initiating the experiment, including sending beginning signals to the receivers and receiving environment variables
	\item Triggering dual data feed via UDP session to the receivers in two Raspberry Pi SoCs
	\item Triggering data collection process in benchmark applications
\end{itemize}
The first feature requires us to be able to send UDP packages to the MPI hosted applications on the cluster before the actual sending process happens, and the initialization should be able to take parameters associate to the experiment we performed. \\
The second feature requires us to be able to start UDP sessions to two different UDP listeners,a nd send packages simulating the A-B data feed discussed before, with fixed offset in latency.\\
The third feature requires us to be able to achieve two things: recording the time stamp in each individual session, and sending the recorded results to the benchmark using TCP sessions, as the data would be too big for the UDP protocol to handle.\\
We uses Business Process Model and Notation to show the flow control of our application\cite{dijkman2011business} in picture \ref{fig:Dualsenderdataflowdiagram-Page1}.

\subsection{Dual-feed handlers and controller}
The dual-feed receiver and controller is running on the Raspberry Pi cluster, consisted of three parts and on the MPI-CH2 framework. It has two feed handers, \text{receiver\_A} and \text{receiver\_B}, and a controller running on the master node.\\
We had two options to design the handlers and controllers over the MPI-CH2, one os using SPMD which only implements one application defining three behaviours, and the \textit{mpirun} take only one applications running on multiple nodes. Behaviours and communications are defined by node rank and tag respectively. \\
Another solution is using MPMD which defines two different applications, one is for the handlers and anther for the controller. Therefore the \textit{mpirun} needs to manage which node implements which application by defining the details in the \textit{machinefile} required by the method.\\
We choose the first solution for three reasons: a united application is easier for coding and testing, and easier in scalability because we can changing the functions of each nodes by reordering the ranks defined in the configuration file of MPI-CH, and the version control over the cluster is much easier than the MPMD solution by using the \textit{rsync} tool.\\
\begin{figure}[H]
	\centering
	\includegraphics[width=1.0\linewidth]{"picture/Dual sender data flow diagram - Page 1"}
	\caption{Dual sender data flow diagram}
	\label{fig:Dualsenderdataflowdiagram-Page1}
\end{figure}
\subsubsection{Requirements for dual-feed handlers}
The main goal of dual feed handlers is receiving data feeds from the source, and seamlessly retransmit the datagrams to the controller for error check. We think about two different approaches:
\begin{itemize}
	\item Synchronize the clock before transmission
	\item Collecting all the datagrams before sending to the controller
	\item Retransmit each datagram to the controller instantly
\end{itemize}
The synchronization of system clock is simulated by setting the initialization point in the runtime using libraries in the Python language, we focus on the solution of consuming the datagrams.\\
We made experiments over the two solutions, and found that the second one affected the overall efficiency of the parallel system, as the instant retransmission will be trigger in the same sequence of the sender, and according to the fat-tree design of the system, the bandwidth of output from the controller is the aggregation of bandwidth over data handlers, therefore the maximum data frequency between the handler and the controller can only be a half of the incoming data.\\
Another analyse was performed by monitoring the verb queues in the MPI-CH architecture, and we found that the queuing buffer requirement for the second solution shows the similar result: the incoming data of frequency $f$, and the average size of the datagram is $s$, the size of verb queue buffer in the controller is approximately $2\times \frac{s}{f}$ with no data loss, because the architecture of the controller is the same as the handlers, therefore the maximum frequency in verb queue consumption equals to that of handers, so the buffer size should be doubled.\\
Along with the distribution strategy of filtering and error check discussed in earlier section, and our choice in the retransmit strategy, we can have our blue print for the dual-feed handlers:
\begin{figure}[H]
\centering
\includegraphics[width=0.5\linewidth]{picture/receiverBlueprint}
\caption{A-B data feed handler blueprint}
\label{fig:receiverBlueprint}
\end{figure}
\subsubsection{Dual-feed handler implementation}
The implementation of dual-feed handler is simple, and A, B handler uses the same structure. We set three fixed variable in the handler: 
\begin{itemize}
	\item \textit{TERMINATE} signal, represents for the end of the session
	\item \textit{timeout} value, we set $1ms$ for each datagram to time out
	\item \textit{keepalive} value, we set $5ms$ for each handler to keep alive and listening to the sender before transforming to waiting mode, and in this mode, the handler will listen to the begin signal from the sender in another port
\end{itemize}
\begin{figure}[H]
\centering
\includegraphics[width=0.9\linewidth]{"picture/Dual-feed handler flow chart - Page 1"}
\caption{Data flow diagram for dual-feed handler}
\label{fig:Dual-feedhandlerflowchart-Page1}
\end{figure}
The initialization of the handler can receive configuration files in MPI structure, synchronized within the cluster, and the end of the handler is when the datagram session is over, and the handler will retransmit the datagram queue to the handler.\\
The two checks: termination check and time-out check ensures that the session will be ended in the two scenarios: the \textit{TERMINATE} signal lost or latency of it exceeds the time-out intervals, or the connection between the handler to the source is cut down for unknown reason, and the session only transmitted a part of the complete message.\\
The implementation of dual data feeds also makes use of these features, as one of the data source may expired abnormally, or both of them do not ended properly, however, there are still chances that the merging result of the two feeds is the original message, and no retransmit from the source will be required.\\
\subsubsection{Controller design}
The controller is the main node in the cluster which aggregates the results from data handlers, and has direct session with the source and the metrics. The handler, as discussed above, mainly focuses on two tasks: datagram filtering. The communication between the source and the controller happens after the verification, if the result shows that the datagrams are corrupted and according to the UDP protocol, a retransmission is required.\\
The design of datagram filtering is simple: the filtering module needs to check the UDP header for checksum and length, and then it can decide whether the datagram is corrupted or not.\\
The controller takes two data feeds, and in general case, data feed A will arrive several milliseconds earlier than feed B, however, in some cases feed B will be faster than A, and one of them will not return any data in certain scenario, as one of the connection from the handlers to source may lost. Therefore we need to make the inter-MPI communication a provider-consumer model, and the controller needs to wait for the verb provider maintained by MPI-CH2 architecture and triggered by the handlers.\\
The RDMA discussed above solves the problem of building the model, as the data marked as shifting from the handler to source will be abstracted into RDMA events by the data driver, and a virtual data bus will serve as the provider to deliver the data to its destination marked as the consumer in the controller. From the view of controller, however, it owns two listeners for the data, and the session will expire after the data bus shifting is over. \\
The blueprint of the controller is showed in picture \ref{fig:controllerBlueprint}. It contains three parts: the data in the data handler which is marked by the RDMA data driver, and a set of RDMA channel, listener and event handler which can perform a seamless data shifting between the handler and the controller using the provider-consumer model. The controller design, however, is simple as the MPI-CH2 architecture abstracts most of the details in RDMA channel communication.\\
In programming level, RDMA seamless communication needs two parameters:
\begin{itemize}
	\item \textit{node\_level} which is defined by the MPI-CH2, and managed by the configuration file which can be manually changed
	\item \textit{message\_tag}, this parameter is used for defining which channel each block of code uses, as the RDMA communication between the nodes can have more than one implementation, and the driver needs to know the exact block of memory it needs to shift to
\end{itemize}
\begin{figure}
\centering
\includegraphics[width=1.0\linewidth]{picture/controllerBlueprint}
\caption{Controller blueprint}
\label{fig:controllerBlueprint}
\end{figure}
\subsubsection{Controller implementation}
As discussed above, we do not need to modify the MPI-CH2 implemented RDMA channels, and the implementation of the controller is consisted of two parts: the implementation of configuration file, the \textit{machinefile}, and the application itself.\\
The structure of \textit{machinefile} shows as follows:
\begin{lstlisting}[breaklines,breakatwhitespace,caption={Example code},label=Example-Code]
Host IP address		Node level
192.168.137.101		2
192.168.137.102		1
...					...
\end{lstlisting}
The IP addresses and node levels are one-one mapping, and no collision of IP address or node level are allowed. The \textit{mpirun} tool will first examine the \textit{machinefile} it points to followed by  the '-machinefile' instructor, and then it will dispatch the applications and RDMA channels according to it.\\
\begin{figure}[H]
\centering
\includegraphics[width=1.0\linewidth]{"picture/Controller data flow diagram - Page 1"}
\caption{Controller data flow diagram}
\label{fig:Controllerdataflowdiagram-Page1}
\end{figure}
The controller application is consisted of two listeners and one verification module containing filtering and datagram error check. We have discussed about how the listener is implemented in the RDMA, and here we focus on how the listener embedded in the overall data flow.\\
The checksum module, however, is implemented in two ways: one is implemented in Python, which runs on CPU platform, and another is implemented over the FPGA discussed earlier using hardware algorithm. The verification module holds the bottleneck of the overall efficiency for the controller, as the decision of whether the session is succeed or not is defined by this module, and downsizing the cost of this module can significantly improve the system efficiency. On the other hand, the FPGA supported application is running in parallel to the main controller, as introduced before, and the controller will be able to listen to the new incoming datagram sessions while the hardware verification program is running, as some datagrams may be oversized.\\
\section{FPGA Verification module design and implementation}
This section introduces how the hardware based verification module is designed and implemented, as the FPGA application uses a different platform, and in order to achieving similar goal as checksum, we need to implement new logic in the hardware different from normal high-level language solutions. The communication between the hardware and software is implemented via SPI over Wishbone, as introduced in last chapter, we will show the details of how it is implemented in the applications.\\
\subsection{Communicator design}
There are two communication strategies for implementing the FPGA verification mode: the first one is that we can introduce a \textit{listener} application running in the memory of the CPU, and listening to the calls from the controller for verification of the data, and fetch the data directly from the memory of CPU platform and slurp it into the FPGA board via Wishbone. The second choice is to implement the hardware communicator in the controller, and the controller invokes a guardian thread to wait for the FPGA verification module to return the results.\\
We tried both and found that neither is applicable:
\begin{itemize}
	\item The first strategy is too complex because the independent listener in the memory needs to synchronize to both the hardware and the controller application, which is almost impossible as the controller will not wait for the verification result and the FPGA do not know the status of the CPU based applications
	\item The second strategy is unachievable because Python multi-threading does not support a return from running thread quite well, and the situation will be unmanageable if the FPGA application died inside the thread. The garbage collection(GC) mechanism in Python is platform depended, and it is too risky to perform GC in a low-latency threading environment\cite{hugunin1997python}
\end{itemize}
Our final implementation of the Wishbone communicator is a combination of the two strategies: no independent thread is needed, as the zero-copy data from the memory to FPGA via SPI is fast enough, and a pause in the main thread for it will not harm the overall efficiency. From the system data flow diagram \ref{fig:Systemdataflowdiagram-Page1}, we can see that the sacrifice in non-parallel verification will not harm the overall design.\\
\subsection{Hardware versification module design}
The input and output of the FPGA application has fixed number of bits, and the frequency is fixed to $50MHz$ by the board\cite{logiwishbone}. Before introducing the design, we need to have a concept of all the restrictions:
\begin{table}[H]
	\centering
	\caption{Logi Pi input/output fixed variables}
	\label{my-label}
	\begin{tabular}{@{}ll@{}}
		\toprule
		Name                    & Description                                                         \\ \midrule
		Input/Output bit number & 16 bits input and output length, split in high 1Byte and low 1Byte \\
		Input/Output address    & 16 bits abstract address, defined in VHDL by Wishbone               \\
		Input/Output frequency  & 50 MHz defined by clock signal on Logi Pi board                     \\ \bottomrule
	\end{tabular}
\end{table}
Therefore, the logic of the algorithm should be based on these restrictions, as we cannot easily put the datagram array into the FPGA and make it calculate the sum of each value, because the machine language do not have these concepts.\\
Another list of concepts we need to have before introducing the design is the hardware modules we need to implement, as discussed in the introduction, the hardware algorithms can be reimplemented and some of the classic design have their standard names\cite{chu2011fpga}:
\begin{itemize}
	\item \textit{master}: the Wishbone master which takes fixed bit-length data from fixed address of SDRAM
	\item \textit{intercon}: the communicator in the master-slave model of SPI architecture, between the master and slaves. The \textit{intercon} differs different slave by addresses
	\item \textit{PWM}: Pulse Width Modulation, we use it to modulate the digital signals and lit the LEDs\cite{westinghouse1957pulse}
	\item \textit{register}: can be viewed as normal logic modules, we can implement simple hardware logics in side, including basic I/O, adders and substractors
\end{itemize}
\subsubsection{Checksum algorithm implementation}
The checksum algorithm needs to consider the following requirements:
\begin{itemize}
	\item The UDP session datagram in practical is close to 512 Byte\cite{deering1998internet}, and maximum to 8192Byte in order to achieve the best practise
	\item Registers can only take 16bits of data as input and output separately
	\item The hardware application runs in parallel to the CPU based application, and the addresses on the FPGA board have no runtime I/O protection, which enables us to perform IO processes without synchronization
\end{itemize}
Therefore, we need up to 4096 registers to perform checksum, or 4096 loops of checksum during the process. If the data cannot be divided by 16, we need to add $0$ at the end of it.\\
We choose to run 4096 loops using one register in the hardware, as it will cost less than $\frac{4096}{5 \times 10^{7} }= 8 \times 10^{-5}$ seconds, far less than the network latency.
The standard checksum process using hardware is as follows:
\begin{itemize}
	\item[1.] Read the pseudo header of the UDP protocol implementations, as it is OS depended
	\item[2.] Pushing the pseudo header, UDP header with the payload of datagram into the FPGA via Wishbone
	\item[3.] The FPGA uses adder to add the data, and keeps the carry
	\item[4.] Add the carry to the final sum, and get the complement code via Wishbone
	\item[5.] Compare the complement code to the checksum
\end{itemize}
\subsubsection{Error check algorithm design and implementation}
The error check algorithm is similar to the checksum algorithm. The FPGA takes the expected length of the data and actual length as inputs in the second register, and perform bit \textit{with} operation to these two numbers, and return the calculation result.\\
The result will be $0x1111$ when correct, and other values will indicate that the error check result is \textit{fail}.\\
\subsubsection{Hardware algorithm design}
The Valent Fx has provided a modular design tool, the Skeleton, over the Logi Pi board, which customized several pre-defined modules that can be implemented over the Logi Pi board\cite{skeleton}.
The design of the hardware algorithm showed in the Skeleton IDE shows as follows:
\begin{figure}[H]
\centering
\includegraphics[width=0.9\linewidth]{picture/skeletonProj}
\caption{Skeleton project design}
\label{fig:skeletonProj}
\end{figure}
The tool will provide us with several fixed addresses for each modules, and the checksum algorithm needs to be implemented in the \textit{reg\_checksum} module. The fixed module addresses are:
\begin{table}[H]
	\centering
	\caption{Static module addresses}
	\label{my-label}
	\begin{tabular}{@{}lll@{}}
		\toprule
		Module Name   & Address & Note                                       \\ \midrule
		PWM\_0        & 0x000B  & PWM connects to the LED                    \\
		PWM\_divider  & 0x0008  & PWM\_divider connects to the system\_clock \\
		reg\_checksum & 0x0000  & Read/Write address of reg\_checksum module \\
		reg\_errorcheck & 0x0011 & Read/Write address of reg\_errorcheck module \\
		 \bottomrule
	\end{tabular}
\end{table}

\section{Conclusion for the Chapter}
This chapter introduces the experiment objectives and describes how each part of the system is implemented. The objective of this experiment is to provide a solution to high performance network using UDP datagram, and introducing the dual data-feed mechanism to reduce the latency in network caused by datagram retransmission.\\
The overall design shows how the data drives each part to perform its functions. The system implementation part focuses on the MPI-CH2 over Raspbian operating system, and how the MPI4Py works. It also introduces some bugs in the libraries and OS, and the way to avoid it. The hardware design also introduced how the system in connected by Ethernet cables, and how the system gains its power supply. The software design formed by three parts: the simulated dual-feed sender, the handler and the controller. Details of how the sender is implemented, and the implementation of intercommunication among the two handlers and controller is provided with detailed metrics, and explained why we choose certain strategy in some parts of the system.\\
The hardware algorithm design introduced in detail about how the checksum mechanism is performed in the FPGA, and we introduced the module structure inside the FPGA by using the IDE desgin diagram of Skeleton.\\
The next chapter will introduce the experiments we implemented, and their analysis.

\chapter{Experiment Requirements, Procedure, and Results}
\begin{figure}[H]
	\centering
	\includegraphics[width=0.7\linewidth]{picture/photo/cluster}
	\caption{The Raspberry Pi cluster framed by Lego}
	\label{fig:cluster}
\end{figure}
This chapter shows the design of experiment, requirements and results of the implemented experiments.\\
The main goal is to implement a data processor for the dual feed data sources using open-source hardware, and implement UDP package processing application on the FPGA for acceleration.\\
In order to achieve the goal of receiving dual feeds, which will come from physically two Ethernet cables, the system needs at least two Ethernet ports, and we introduces MPICH over multiple independent hardware which has one Ethernet port each, and congregate them into a cluster in order to meet the requirements.\\
\newpage
\section{Experiment Design}
This section introduces the design of three groups of experiments and their architectures:
\begin{table}[H]
	\centering
	\caption{Experiments}
	\label{my-label}
	\begin{tabular}{@{}|l|l|l|@{}}
		\toprule
		Number & Name                                                                                                   & Introduction                                                                                                                                                                  \\ \midrule
		1      & Single data feedwith single receiver                                                                   & Normal UDP send/receive pair                                                                                                                                                  \\ \hline
		2      &\begin{tabular}[c]{@{}l@{}} Dual data feeds over the cluster\\ with MPI-CH2 and FPGA\end{tabular}                                                                       & \begin{tabular}[c]{@{}l@{}}Using dual data feeds of normal \\ UDP send/receiver pairs accelerated\\ via FPGA\end{tabular}                                                                    \\ \hline
		3      & \begin{tabular}[c]{@{}l@{}}Dual feeds processed with MPI \\ processors without FPGA\end{tabular}       & \begin{tabular}[c]{@{}l@{}}Using dual data feeds sending data\\ to MPI structure receivers without FPGA\end{tabular}                                                          \\ \bottomrule
	\end{tabular}
\end{table}
For each groups of experiment, three basic processes will be performed: getting the basic network latency, get the network latency with data loss, and checking the performance of the system if hazard situation happens, especially when one of the data sources is lost.\\ 
The goal of implementing these experiments is to make comparisons of the network system with and without the dual-feed topology, and compare the open-sourced dual-feed system with the commercial ones, and see if the open-source system can achieve similar acceleration compared to the commercial ones.\\

\subsection{Performance metrics}
The performance metrics of the network can have multiple definitions, and we made a framework for the metrics to focus on the efficiency in the following terms: the mean time for a success session, the average success ratio within certain time interval and overall success ratio in extreme conditions\cite{cerf1991guidelines}.\\
There are two different metrics system: the analytical metrics and empirical metrics\cite{paxson1996towards}. Discussed in Paxon`s paper, the empirical metrics can only be measured via program, and analytical ones can be derived from the benchmark results.Therefore, we use empirical metrics in the experiment for a direct comparison in network efficiency, and compare to the 10Gb Ethernet solution by using the analytically metrics derived from the experiment results.\\
\subsubsection{Basic terminologies}
We need to introduce some basic terminologies from a general framework for network performance benchmark before the next step discussion, and these terminologies are not related to certain way of platform implementation\cite{paxson1996towards}:
\begin{itemize}
	\item \textbf{Host}: a computer linked to the network
	\item \textbf{Cluster}: in this project, a cluster is specifically refers to the Raspberry Pi cluster, consisted of three Raspberry Pi hosts, and can be treated like a \textit{black box}
	\item \textbf{Network Path}: an virtual path between two hosts, exists in the Internet Layer  in TCP/IP model
	\item \textbf{Route}: specifically how the hardware devices, the hosts, linked to form a network path
\end{itemize}
\subsubsection{Empirical metrics}
As discussed above, we need a framework of empirical metrics in the experiment:
% Please add the following required packages to your document preamble:
% \usepackage{booktabs}
\begin{table}[H]
	\centering
	\caption{Empirical metrics}
	\label{my-label}
	\begin{tabular}{@{}lll@{}}
		\toprule
		Name                                                             & Unite          & Note                                                                                                       \\ \midrule
		Transmission efficiency                                                & second per session & \begin{tabular}[c]{@{}l@{}}The time cost for one times of \\sessions from host A to B\end{tabular} \\
		\begin{tabular}[c]{@{}l@{}}Transmission\\ frequency\end{tabular} & per session    & \begin{tabular}[c]{@{}l@{}}The reciprocal of transmission times \\ for a success session\end{tabular}      \\
		Datagram received                                                 & per session    & 
		\begin{tabular}[c]{@{}l@{}}The correct datagram received in one\\ session        \end{tabular}                                                        \\ \bottomrule
	\end{tabular}
\end{table}
The transmission time refers to the cost in datagram transmission over the network path in the system, and it will be affected by the checksum mechanism, because 'corrupted' datagrams will cause a retransmission of the entire session.\\
The transmission frequency refers to the situation when the data received is 'corrupted', and the sender needs to retransmit the data. The number of times it take to have a success session can be an overall metrics of system efficiency.\\
The datagram received refers to the number of correct package received in a single session. This number will be affected by data lost and checksum error, as retransmission will add a new datagram to it.
\subsubsection{Analytical metrics}
We also implement some analytical metrics to compare the experiment hardware to the commercial ones:
% Please add the following required packages to your document preamble:
% \usepackage{booktabs}
\begin{table}[H]
	\centering
	\caption{Analytical metrics}
	\label{my-label}
	\begin{tabular}{@{}lll@{}}
		\toprule
		Name               & Unit                & Note                                                                                                      \\ \midrule
		Acceleration ratio & \%                  & Latency with dual-feed against latency without it                                                         \\
		Bandwidth cost     & Kilo bit per second & \begin{tabular}[c]{@{}l@{}}Derived bandwidth cost of the system implemented\\ with dual-feed\end{tabular} \\
		Power cost         & Kilo Walt           & The energy cost of the system                                                                             \\ \bottomrule
		\end{tabular}
\end{table}
\subsection{Choice on inputs}
The inputs of the experiments are UDP datagrams, however, the frequency of UDP datagram and the length of the data inside the datagram should be chose carefully, as the input of the experiment should be able to simulate the real world situation.\\
According to the discussion of data loss ratio in the above chapter, the data loss ratio is the same as the real world, and according to the expression of transmission time cost $T_{trans}=(1+P_{lost})\times T$, T is the one-side average transmission time without data lost. In order to compare the acceleration ratio of the experiment $Acc_{exp}$ with the real world one $Acc_{real}$, we have the term expected acceleration compared to te established one:
\begin{equation}
	E_{acc}=\frac{{T}'_{trans}}{T_{trans}\times Acc_{real}}
\end{equation}
The ${T}'_{trans}$ is the transmission time in dual-feed system with data loss, and $T_{trans}$ is the one without dual-feed mechanism. It is affected by both the data size of datagrams and frequency of the sessions, however, the size of the data 
\subsection{Experiment environment set up and pre-processes}
After booting up the MPI-CH2 environment on the cluster, we need to make the following setups before the experiments:
\begin{itemize}
	\item Disable the UDP checksum mechanism in both the server host and cluster
	\item Synchronize the application repository in the cluster using \textit{rsync} command
	\item Open the \textit{Jupyter Notebook} in the server host to record the metrics using \textit{iPython} 
\end{itemize}
The first step is to ensure the system based checksum is disabled, as we need to perform checksum by the applications, and the NIC card in both side should not be able to perform the checksum of the payload.\\
We have set up three empirical metrics and three analytical metrics, and the next section will describe the process of each experiment, and the results of them.

\subsection{Experiment set 1: Single data feed with single receiver}
This experiment shows the behaviour of a single UDP sender(data source) and a single UDP receiver(client), and tracks the output of receiver in the following environment:
\begin{itemize}
	\item No data lost in source and transmission
	\item Data lost in source
	\item Significant latency in data transmission
	\item Data source is down
\end{itemize} 
This experiment requires a host simulating the data source, and a raspberry pi hosted receiver as the client. Quality of service strategy in UDP datagram, which is the checksum of datagram is relied on the default settings of the client operating system.\\
\begin{figure}[H]
\centering
\includegraphics[width=0.5\linewidth]{picture/Exp1}
\caption{Design of experiment 1}
\label{fig:Exp1}
\end{figure}

\subsubsection{Experiment results and analysis}
This experiment aims at figuring out the basic environment empirical metrics under the traditional network topology, with single feeding data, and we made the processes discussed above.
\subsubsection{1. No data lost between source and transmission, low frequency data}
\subsubsection{Inputs}
\begin{table}[H]
	\centering
	\caption{Inputs for experiment 1.1}
	\label{my-label}
	\begin{tabular}{@{}llll@{}}
		\toprule
		Input          & Unit & Value           & Note                           \\ \midrule
		UDP datagram   & packet/session & 1     & The sessions without data loss \\
		UDP session & per second   & 500         &                                \\
		Data loss rate & \%   & 0         &                                \\
		 \bottomrule
	\end{tabular}
\end{table}
We made 1000 independent session in sequence.
\subsubsection{Results and analysis}
\begin{figure}[H]
\centering
\includegraphics[width=0.6\linewidth]{picture/experiments/exp1/latency}
\caption{Latency with no data lost}
\label{fig:latency}
\end{figure}
The average transmission time between the sender and receiver is 39.655ms. 
% Please add the following required packages to your document preamble:
% \usepackage{booktabs}
\begin{table}[H]
	\centering
	\caption{Empirical metrics results for experiment 1.1}
	\label{my-label}
	\begin{tabular}{@{}lll@{}}
		\toprule
		Metrics                 & Unit               & Result                 \\ \midrule
		Transmission efficiency & second per session & $3.9655\times 10^{-2}$ \\
		Transmission frequency  & per second        & 50(fixed)              \\
		Datagram received       & number per session & 1(fixed)            \\ \bottomrule
	\end{tabular}
\end{table}

\subsubsection{2. Data lost in source, the data lost ratio is 0-3.4\%}
In this experiment, we have set the time-out value to $1$ second, as some of the retransmission will take multiple times. For each data node, it represents the time cost $y$ in having $x$ amount of success UDP datagrams-length session between the sender and receiver in high frequency. However, the average time cost raises significantly as the amount of datagrams rises, compared to the situation when no data lost happens, the average network latency turns out to be very stable. The reason for it is that, the data loss of the UDP package will trigger a full retransmission, and the increasing amount of datagram retransmitted largely impacts the overall latency performance.\\
\subsubsection{Inputs}
\begin{table}[H]
	\centering
	\caption{Inputs for experiment 1.2}
	\label{my-label}
	\begin{tabular}{@{}llll@{}}
		\toprule
		Input          & Unit           & Value  & Note                     \\ \midrule
		UDP datagram   & packet/session & 0-990 & Increasing by 10 per set \\ \hline
		UDP session    & per second     & 500    & \begin{tabular}[c]{@{}l@{}}The next session will not start\\ until the previous one finishes  \end{tabular}\\ \hline
		Data loss rate & \%             & 0-3.4  & Random for each session  \\ \bottomrule
		\end{tabular}
\end{table}
We made 100 sets of tests, and measure the mean transmission time between the sender and receiver.
\subsubsection{Results and analysis}
The result of this experiment is in picture \ref{fig:dataLoss}, however, the accurate empirical result is hard to be obtained in this experiment for the following reasons:\\
First, we introduces the data loss ratio instead of simulating the real data loss, and the UDP protocol does not provide CRC in the header. We cannot inspect the 'corrupted' diagrams, even though this mechanism increases the overall transmission speed with smaller header. It also makes the number of 'corrupted' diagrams to increase as the distribution of wrong 'bits' will be flattened over the diagram as the amount of samples increases. The system is sensitive to the randomized data loss rate.\\
Second, the time-out mechanism makes the measurement result inaccurate, as all the time-out results are set to be 1000 ms.\\
The empirical results of average transmission time varies from 49ms to 54ms, and the average datagram received is 92-96\% of the sent datagrams. However, the result of this experiment can not be compared to the general results, as the data is sensitive to the actual data loss ratios, and we will have analytical results instead.\\
\begin{figure}[H]
	\centering
	\includegraphics[width=0.7\linewidth]{picture/experiments/exp1/dataLoss}
	\caption{Experiment result with data loss}
	\label{fig:dataLoss}
\end{figure}
\subsubsection{K-S test for the randomized data loss ratio}
We implements the \textit{Scipy} package in Python to test if the data loss happens follows the Gamma distribution\cite{scipy}. We generate the random integer list with the \textit{numpy} library using $mean=3.0435$, $dispersion \phi =2$, and the significant test passes with the value pair $D=0.00363, \phi = 0.1342$. The distribution of data lost fit the distribution pattern of Gamma distribution.\\

\subsubsection{3. Data source lost connectivity}
\subsubsection{Inputs}
\begin{table}[H]
	\centering
	\caption{Inputs for experiment 1.3}
	\label{my-label}
	\begin{tabular}{@{}llll@{}}
		\toprule
		Input          & Unit & Value           & Note                           \\ \midrule
		UDP datagram   & packet/session & 1     & The sessions without data loss \\
		UDP session & per second   & 500         &                                \\
		Data loss rate & \%   & 0         &                                \\
		\bottomrule
	\end{tabular}
\end{table}
The input is similar to experiment 1.1, however, we manually unplug the connection simulating an unexpected loss of connectivity.\\
\subsubsection{Results and analysis}
The live diagram of this experiment is missing, as the \textit{metaplotlib} requires a full session to parsing the data structure. We managed to get the diagram from the csv file we created during the session.\\
The value 0 means the CSV writer cannot catch new data, and write a zero to the result after time-out.
\begin{figure}[H]
\centering
\includegraphics[width=0.7\linewidth]{picture/experiments/exp1/lostConnect}
\caption{Single feed lost connection}
\label{fig:lostConnect}
\end{figure}
\subsection{Experiment set 2: Dual data feeds over the cluster on MPI-CH2 with FPGA}
This experiment shows the advantage of using dual feeds. Data source needs to simulate two data sources with the same content in package, and has an manual offset of one millisecond. The experiment needs to track the output from receivers under the following scenarios:
\begin{itemize}
	\item No data lost in either sources
	\item One of the source simulates data lost, while the other does not
	\item Both sources have data lost
	\item One of the sources goes off-line, while the other one does not
	\item Both sources go off-line
\end{itemize}
The design is shown in picture \ref{fig:exp3}.\\
Receiver 1 and 2 can communicate with each other using network protocols and contact $output_{1}$ and $output_{2}$ together to have an unified output, and shows the differences in data lost conditions.\\
The checksum mechanism is also relied on the configurations in operating system of the receiver, and the merging process will happen in the simulated master node.\\ 
\begin{figure}[H]
	\centering
	\includegraphics[width=0.8\linewidth]{picture/exp3}
	\caption{Design of experiment set 2}
	\label{fig:exp3}
\end{figure}
\subsubsection{Experiment results}
\subsubsection{1. No data lost in either sources}
\subsubsection{Inputs}
\begin{table}[H]
	\centering
	\caption{Inputs for experiment 2.1}
	\label{my-label}
	\begin{tabular}{@{}llll@{}}
		\toprule
		Input          & Unit & Value           & Note                           \\ \midrule
		UDP datagram   & packet/session & 1     & The sessions without data loss \\
		UDP session & per second   & 500         &                                \\
		Data loss rate & \%   & 0         &                                \\
		Feed offset & millisecond   & 5         &                                \\
		\bottomrule
	\end{tabular}
\end{table}
Both A feed and B feed follows the same metrics, and the offset between the two is 5 milliseconds.
\subsubsection{Results and analysis}
\begin{figure}[H]
\centering
\includegraphics[width=0.7\linewidth]{picture/experiments/exp2/dualLatency}
\caption{Dual data feed network latency without data loss compares to the one without data loss}
\label{fig:dualLatency}
\end{figure}
The average transmission time cost is 41.0039982666ms. This result shows the system works as we expected, because the latency of network does not change significantly even we added a handler layer between the actual receiver and the sender. The network latency using RDMA intercommunication is much more efficient than the UDP sockets.\\ 
\begin{table}[H]
	\centering
	\caption{Empirical metrics results for experiment 2.1}
	\label{my-label}
	\begin{tabular}{@{}lll@{}}
		\toprule
		Metrics                 & Unit               & Result                 \\ \midrule
		Transmission efficiency & second per session & $4.1003\times 10^{-2}$ \\
		Transmission frequency  & per session        & 50 each(fixed)              \\
		Datagram received       & number per session & 1000 each(fixed)            \\ \bottomrule
	\end{tabular}
\end{table}
\subsubsection{2. One of the source simulates data lost, while the other does not}
We made a comparison of this result with the one only has one data feed. 

\subsubsection{Inputs}
\begin{table}[H]
	\centering
	\caption{Inputs for experiment 2.2}
	\label{my-label}
	\begin{tabular}{@{}lllll@{}}
		\toprule
		Feed & Input          & Unit           & Value  & Note                     \\ \midrule
		& UDP datagram   & packet/session & 10-990 & Increasing by 10 per set \\
		A    & UDP session    & per second     & 500    &                          \\
		& Data loss rate & \%             & 0      &                          \\ \midrule
		& UDP datagram   & packet/session & 10-990 & Increasing by 10 per set \\
		B    & UDP session    & per second     & 500    &                          \\
		& Data loss rate & \%             & 0-3.4  &                          \\ \bottomrule
	\end{tabular}
\end{table}
\subsubsection{Results and analysis}
The comparison of overall transmission delay shows in picture \ref{fig:oneDataLost}.\\
\begin{figure}[H]
\centering
\includegraphics[width=0.7\linewidth]{picture/experiments/exp2/oneDataLost}
\caption{With B feed data lost compared to single data feed}
\label{fig:oneDataLost}
\end{figure}
The comparison between the two situation shows the efficiency of the dual data feeds is higher than that with one data feed, as the data loss will not affect the system, as the controller will always have the correct result, because one of the sources have fixed zero data loss.\\
In this scenario, the average transmission time per session does not change significantly compared to the previous experiment, and the improvement in network latency performance is more significant when the traffic is busy, as is shown on the right part of the diagram: the red line of single data feed has time-out, however, the green one with dual data feeds performs as normal.\\

\subsubsection{3. Both data feeds are tuned with data loss}
\subsubsection{Inputs}
\begin{table}[H]
	\centering
	\caption{Inputs for experiment 2.3}
	\label{my-label}
	\begin{tabular}{@{}llll@{}}
		\toprule
		Input          & Unit           & Value  & Note                     \\ \midrule
		UDP datagram   & packet/session & 0-990 & Increasing by 10 per set \\ 
		UDP session    & per second     & 500    & 							\\
		Data loss rate & \%             & 0-3.4  & Random for each session  \\
		Feed offset & millisecond   & 5         &                                \\ \bottomrule
	\end{tabular}
\end{table}
Both A and B data feeds follows the same input metrics, the offset between the two is 5 millisecond.\\
\subsubsection{Results and analysis}
This experiment set A-B data feed both with random data loss, and the controller cannot guarantee if it can receive the correct datagram. If the merging result still has data loss, the controller will trigger an \textit{resend} signal to the source, and both A and B data feed will retransmit the datagrams of the same session.\\
We made a comparison of the three scenarios all with data loss, and the result shows in picture \ref{fig:twoDataloss}.

It is worth highlighting that even though the latency raise significantly compared to the one with reliable source, it still does not reach the time-out, and only raise to less than half of the situation when there are significant data loss in the network. The experiment result is stable and not sensitive to the data loss, therefore we have the empirical results:\\
\begin{table}[H]
	\centering
	\caption{Empirical results for experiment 2.3}
	\label{my-label}
	\begin{tabular}{@{}lll@{}}
		\toprule
		Metrics                 & Result                              & Note                                                                                 \\ \midrule
		Transmission efficiency & $4.936\times 10^{-2}s/session$      &                                                                                      \\
		Transmission frequency  & 5 - 900 datagrams/second            &                                                                                      \\
		Datagram received       & approximately 496 datagrams/session & \begin{tabular}[c]{@{}l@{}}This value is sensitive \\ to data loss rate\end{tabular} \\ \bottomrule
	\end{tabular}
\end{table}
\begin{figure}[H]
	\centering
	\includegraphics[width=0.7\linewidth]{picture/experiments/exp2/twoDataloss}
	\caption{Comparison of three data-loss scenarios}
	\label{fig:twoDataloss}
\end{figure}
\subsection{Experiment set 3: Dual feeds processed with MPI processors, without RDMA and FPGA separately}
This experiment implements MPICH over the Raspberry Pi2 clusters, and the experiment shows output on the master node with dual data feed sources under the same scenarios as experiment2. The receiver programme communicate with each other using the message passing interfaces provided by MPICH, and shows the details of latency handling.\\
The MPICH program uses MPI\_recv and MPI\_send pair to communicate under the MPI framework, and the way it runs is using SSH client in the master to perform \textit{mpiexec} shell command, with the arguments of a file containing lists of servers, and configurations of pointing at the destination of each MPI programme. The intercommunication can be implemented with or without RDMA, and this experiment aims at comparing the differences between the two choices.\\
We also made two versions of controller, with and without the acceleration of FPGA hardware algorithm. The comparison between the two is expected to show us the necessity of implementing the FPGA hardware.\\
\begin{figure}[H]
	\centering
	\includegraphics[width=1.0\linewidth]{picture/experiments/exp3/delay_overall}
	\caption{Comparison with other experiment results}
	\label{fig:delay_overall}
\end{figure}
We design the two subsets of experiments:the hardware environment of subset one is using FPGA without RDMA
\begin{itemize}
	\item Cluster receiving data feed without data loss
	\item Cluster receiving data feed with both feed data loss
	\item Cluster receiving data feed with hazard situation when sources off-line unexpected
\end{itemize}
The hardware environment of subset one is using RDMA without FPGA, and the experiment processes are the same.We expected to see a significant reduce in system performance when the RDMA or FPGA hardware are removed, however, the experiment results told a different story in \ref{fig:delay_overall}.\\
From the result diagram, we can see that all three metrics are similar to the results in experiment 2, and the overall performance of the system is not sensitive to the implementation of FPGA and RDMA. However, we searched for the materials in Beowulf cluster analysis\cite{sterling2002beowulf}, and low latency network infrastructures\cite{thekkath1993limits}, both mentioned one important bottleneck of system efficiency, which is the checksum mechanism over the payload, especially in IPv4 environment.\\
They believed that the checksum performed by NIC card following the RFC standards is not efficient for high performance network devices, however,in the pre-process, we have disabled the UDP payload checksum.\\
It is necessary to add an additional experiment which enables the checksum mechanism, because the reason for us to implement the FPGA and RDMA channel is to substitute the filtering mechanism in NIC cards, and in the UDP datagram transmission, the filtering is the checksum mechanism.\\
\subsection{Additional Experiment: Dual feeds processed with MPI processors}
Back to the discussion in the first chapter, we have listed some bottlenecks which may severely affects the latency performance of the system. This set of experiment is trying to figure out if the checksum mechanism in the NIC card will convincingly enough to affect the system performance.\\
\begin{figure}[H]
	\centering
	\includegraphics[width=0.7\linewidth]{picture/addExperi}
	\caption{Two parts of measurement}
	\label{fig:addExperi}
\end{figure}
\subsubsection{1. Get zero retransmit latency}
The experiment implementation is different from the previous three ones, however, as the checksum now happens in NIC card, we first need to find the transmission time without retransmit for error bits, then make a comparison between the network latency with and without NIC checksum respectively.\\
We constructs raw UDP packets with checksum equals to zero to avoid the NIC automatic mechanism, however, as the structure of the cluster uses two Raspberry Pi hosts as pure data feed handlers, we need to calculate the latency in to parts showed in picture \ref{fig:addExperi}.\\
We first measure the latency between the sender and handler when both the NIC cards in handlers have UDP checksum enabled.\\
\begin{figure}[H]
	\centering
	\includegraphics[width=0.7\linewidth]{picture/experiments/exp4/latency}
	\caption{Network latency with NIC checksum turned on}
	\label{fig:latency}
\end{figure}
It is easy to observe that transmission time cost from the handler to sender is 38ms. Our implementation for the interconnection within cluster does not use UDP protocol, as discussed in the previous chapter, and we believe the latency in intercommunication is much lower than that between the sender and handler.\\
The checksum mechanism does not affect the system efficiency when no data loss in the transmission with low frequency data, and we continue our experiment with high frequency data with data loss.\\
The checksum mechanism does not influence UDP sessions within the cluster, therefore we can assume that the latency from the receiver to sender is limited by the connections between sender and handler.
\subsubsection{2. Both feeds with data loss and checksum in NIC turned on}
We turned on the checksum mechanism in both NIC cards of the handler, and measure the transmission time between the sender and receiver. We implement this experiment to find out whether the checksum mechanism will influence the latency of the network. The results in experiment 4.1 proved that the checksum in NIC card would not affect data transmission between the sender and handler in traditional single data feed with no data loss.\\
\subsubsection{Inputs}
\begin{table}[H]
	\centering
	\caption{Inputs for experiment 4.2}
	\label{my-label}
	\begin{tabular}{@{}llll@{}}
		\toprule
		Input          & Unit           & Value  & Note                     \\ \midrule
		UDP datagram   & packet/session & 0-990 & Increasing by 10 per set \\ 
		UDP session    & per second     & 500    & 							\\
		Data loss rate & \%             & 0-3.4  & Random for each session  \\
		Feed offset & millisecond   & 5         &                                \\ \bottomrule
		\end{tabular}
\end{table}
Both A source and B source follow uses the same parameters.\\
\subsubsection{Results and analysis}
\begin{figure}[H]
\centering
\includegraphics[width=1.0\linewidth]{./picture/experiments/exp4/dataLoss_compare.jpg}
\caption{Results for experiment 4.2}
\label{fig:dataLoss_compare}
\end{figure}
The result shows that the latency between sender and receiver will exceed the time-out limit in the receiver side, and in comparison with the results in experiments set 3, we can find out the reasons that cause this problem:\\
The checksum mechanism in NIC card lower the efficiency between the handler and the sender, and the datagram will be retransmitted several times before being retransmitted to the receiver. Compared to experiment set 3, whose datagram checksum happens in the receiver, and the datagrams are merged after the checksum, and the 'corrupted' datagrams will be ignored if anther feed has the correct one, and no retransmission will be performed in such scenario.\\
However, according to the data loss ratio we set, the overall data loss ratio should be $1-0.966^{2}=6.68\%$, which is much lower than the empirical results. We guess it is because that the checksum mechanism is not parallel for the channel between the NIC cache to the system memory, and the channel will wait for the checksum to tag the correct datagram before slurping into the user level memory spaces. This could be the bottleneck for the system, as the data filtering will either exceed the time-out limit between the handler and receiver, or the amount of received datagram would flood the NIC cache, and the data loss ratio will increase significantly in high-frequency transmission environment.\\
The results also show that the speed of filtering can influence the system efficiency especially when we implement the high frequency data, as the curve reaches time-out limit more when the frequency is higher than 400 packages per session.\\
\section{Analysis and Conclusion}
We have made four sets on experiments, the first three ones are performed with checksum in receiver, the fourth one equipped checksum in the NIC card in handlers.\\
The first set shows the latency in traditional single feed between sender and receiver is 39.655ms, and the latency will significantly increase to larger than the 1000ms time-out limit when the sender increases the transmission frequency with random data loss.The single feed solution also cannot resolve the hazard situation when the data source lost connection.\\
The k-s test shows that our settings in $p-value$ and $D$ in the distribution of data loss fits the Gamma distribution.\\
The second set shows the situation when implementing dual data feeds. The latency between sender and receiver is 41.003ms, which is close to the latency in traditional solution. However, the empirical results of high-frequency data with data loss are much more improved than the traditional one, as dual-feed sources reduce the retransmission time by introducing merging mechanism between checksum and error check. This notably reduces the network latency by 70-80\% when one of the source is reliable with no data loss, and 50\% when both sources are not reliable. The dual-feed mechanism can also prevent 100\% data loss when one of the source lost connection, however, the metrics should be the same as the single-feed one with data loss.\\
The third one tried to measure the efficiency in FPGA hardware algorithm and RDMA channel, however, we find that these two implementation do not influence the system performance as we expected.\\
We implemented the fourth experiment set to prove the necessity of implementing the FPGA and RDMA, and this experiment shows that the system is not usable when we implement the UDP checksum in the NIC card, especially when we increase the data frequency. The cause can be split into two parts:
\begin{itemize}
	\item The checksum in NIC of the handlers requires both feed with no bit error, therefore the loss ratio increases to 6.68\%
	\item The cache memory space in NIC of the handlers in commodity hardware is not big enough to handle high-frequency data while the filtering mechanism is single threaded to the content channel between cache and user memory space
\end{itemize} 
It is necessary to disable the NIC payload filtering mechanism, according to this experiment, and our analysis result shows that acceleration in FPGA filtering can be more conspicuous if we higher the data frequency.\\
The RDMA channel is also necessary because the transmission with pure MPI communication interfaces is less efficient than the RDMA solution, as explained in the previous chapter, and the direct data swapping is lower in both time and memory space.
\chapter{Conclusion and future work}
This chapter aggregates the results from the previous chapter, make analysis and draw the conclusion about this project.\\
The analysis focus on the following aspects:
\begin{table}[H]
	\centering
	\caption{Aspects in this chapter}
	\label{my-label}
	\begin{tabular}{@{}l|l@{}}
		\toprule
		Aspect               & Description                                                                                                                                                                    \\ \midrule
		Main contribution            & \begin{tabular}[c]{@{}l@{}}The main goal of this project is to establish an open-source\\ dual-feed handler, and run the experiments to prove that it is\\ usable\end{tabular} \\ \hline
		Challenges     & \begin{tabular}[c]{@{}l@{}}There are some existing problems to be solved, and leading to the future\\ work\end{tabular}                                                        \\ \hline
		Future work & \begin{tabular}[c]{@{}l@{}}This project is expected to be applicable using 10Gb Ethernet hardware,\\ more features and tests will be performed\end{tabular}                \\ \bottomrule
	\end{tabular}
\end{table}
\newpage
\section{Main Contributions}
The overall design of the system is inspired by the articles about establishing Beowulf clusters using heterogeneous hardware\cite{kiepert2013creating}, and established network optimization using two UDP data feeds\cite{chordia2013high}. \\
As illustrated above, we have built a small cluster made of three piece of open-source hardware, the Raspberry Pi 2, and simulated the high frequency dual data feeds using UDP sockets, and the data handler running on two Raspberry hosts works with the controller node, successfully consumed the data. The experiment results showed that the cluster worked as expected, and reduce the network latency compared to the traditional network topologies. The hardware system is encapsulated using Lego building blocks to achieve fine cabling layouts and scalability.\\ 
\begin{figure}[H]
	\centering
	\includegraphics[width=0.9\linewidth]{picture/photo/fourScene}
	\caption{Raspberry Pi cluster from four angles}
	\label{fig:fourScene}
\end{figure}
\subsection{Hardware}
The network cabling and Lego built cabinets provides the cluster with portability, and four standard 5V power supply can be supported by either power strips or portable batteries. The USB cables supports peripherals such as keyboard, mouse and wireless network card. The HDMI output of the cluster support resolution up to 1400x960 according to the manual\cite{upton2014raspberry}.\\
\subsection{Software system}
The cluster is running on the Raspbian, which is a customized version of Debian operating system. The hosts in the cluster is interconnected by a switch turned into broadcast model. Applications running on the hosts within the cluster communicate to each other via MPI-CH2 framework, the implementation of Message Passing Interface. We also implement the Remote Memory Direct Access channel in the cluster, therefore the user level application memory contents can be swapped seamlessly by the data driver and channel provided by the RDMA module. The cluster can be accelerated using two methods: the Overclock boost on the Raspberry Pi board\cite{upton2012meet}, or using the FPGA connected to the mastser node via General-purpose input/output ports. \\
The cluster takes inputs via Secured Shell(SSH), and we host a svn repository in the main developing areas, and users can access to the source codes easily.\\
\subsection{Data sender}
The sender is a simulator that can send UDP datagrams in different model. The models includes:
\begin{itemize}
	\item Sending data feed to single source with/without data loss
	\item Sending data feeds to dual data sources with manual offset in latency. The UDP datagram checksum=$0x0000$ and using TCP sessions to transmit correct checksum, and collect metrics
	\item Sending data feeds to dual data sources with manual offset in latency, the UDP datagram with normal checksum 
\end{itemize}
\subsection{Dual-feed data handler}
The main part of software implementation is the dual-feed data handler, which consumes UDP datagrams, and the cluster works as a high performance Internet switch particular for the UDP data transmission. This application system is implemented using Python version 2.7.3.\\
The dual-feed handler is distributed on all three nodes, and follows the structure of Multiple Process Multiple Data(MPMD), the data handler code block is  running on two hosts, and a controller is running on the master node. A hardware algorithm in the FPGA is running to accelerate the checksum process for the UDP datagram queues.\\
The UDP payload checksum is disabled in kernel model, and users of the handler application needs to construct their own raw UDP datagram. Data handler A receives data from IP 192.168.137.101 with port 3333, and data handler B receives data from 192.168.137.102 with the same port. The data handler can retransmit received datagrams following a customizable list of hosts with IP addresses, ports and hostnames.\\
The handler only supports connection from RJ-45 cables at present due to the limitation in hardware portals. This will be discussed in project expectations.\\
\subsection{Receiver}
The receiver consumes the retransmission data from the data handlers, and implemented with Python 2.7.3.\\
The receiver receives data feeds from the handlers using MPI communication channels, and merge the sources into a single result, and perform error check to the result. The filtering of checksum happens when the data is transmitted via the MPI channels, and 'corrupted' datagrams will be thrown by the mechanism. Merging mechanism will merge both feeds, and the error check will examine the merged result to see if a retransmission is required.\\
The receiver is attached to an FPGA in hardware, and the checksum and error check both have two versions: the pure Python one and FPGA hardware accelerated one.\\
\section{Challenges}
The project is an experimental platform, which proves that the commodity hardware can implement the high-frequency data feed infrastructure by forming a Beowulf cluster with the help of Message Passing Interface.\\
There are some major unsolved problem in this platform:\\
\begin{itemize}
	\item No back-up power supply for the system
	\item The system is connected with the RJ-45 cables, whose metrics is much lower than the 10Gb Ethernet cables
	\item The experiment platform does not have relays between the sender and handlers, and the influence of the relays are not considered
	\item The Beowulf cluster is consisted of three nodes, which is much smaller than an applicable cluster
\end{itemize}
The cluster is supported by a power strip connecting to the public electric grid, with no backup or hazard tolerance mechanism, which is an unacceptable problem if we want to implement the solution in working environment.\\   
The established solutions are implemented over the 10Gb Ethernet hardware infrastructures, and we need to test our solutions under the same environment if we want to substitute these commercial equipments with the commodity hardware, which is our final goal.\\
The real-world scenario of high performance network usually connects each other from long distance, and the behaviour of relays between the nodes can influence the network performance.\\
We implement the cluster with three heterogeneous nodes connected with wired cables, which is limited by the resources we are provided. However, the system performance could be improved by implementing more computational nodes, especially in the receiver which is performing a single threaded parser.\\
\section{Future work}
This section lists the components we expect to add to the project in the next phase, the additional experiments related we expect to perform and implementations we want the project to be in the future.\\
\subsubsection{Unstoppable Power Supply(UPS)}
We expect to be able to implement an Unstoppable Power Supply between the direct power feed and the system, which will increase the availability of the system.\\
We have designed a UPS in the system, however, the special requirement for the cluster makes it hard to implement an UPS, because the Raspberry Pi boards are supported by the USB cables and the switch is supported by the 3.5mm power adapter.\\
We expect to find an UPS hardware which supports a mixed power outputs.\\
\subsubsection{10Gb Ehternet port for commodity hardware}
We have found the established USB-based Gb Ethernet cable support for the implemented commodity hardware, for example the D-link USB to GB Ethernet solution shows in picture \ref{usbtogb}.\\
\begin{figure}[H]
	\centering
	\includegraphics[width=0.5\linewidth]{picture/usbtogb.jpg}
	\caption{USB to Gb Ethernet port by D-link}\cite{dlink}
	\label{fig:usbtogb}
\end{figure}
However, we only find the SoC solutions for the 10 Gb Ethernet ports, which requires additional interconnection structure and power supply. We will keep on with the hardware implementations of a portable USB based 10 Gb Ethernet port solution.\\
\subsubsection{Comparison between the MPI solution to the GPIO solution}
We have analysed the choice between the implemented MPI solution and the GPIO based solution, which interconnects the system via SPI interfaces. However, we should compare the efficiency of interconnection using the GPIO system. The GPIO solution sacrifice the scalability for communication efficiency, as we analysed, and we need to experiment over at least two sets of inputs: the small amount of inputs which the GPIO can retransmit within the chip clock period, and the large one which exceeds the clock period, and requires multiple times of retransmission.\\
\subsubsection{Comparison with the real implementation}
We compare the results with the established system using the provided analytical results from the reports, however, we should compare the real-time metrics between the commodity cluster with the commercial solutions after implementing the 10-Gb solution on the commodity one, and make further improvements on the system.\\  
\bibliography{ExampleBibFile}



\appendix

%%
%% Use the appendix for major chunks of detailed work, such as these. Tailor
%% these to your own requirements
%%

\chapter{Definition dispersion parameter $\phi$ of Gamma distribution}
For the random value $Y$ following Gamma distribution $Gamma(\alpha,\nu)$, the density function is\cite{lukacs1955characterization}:\\
\begin{equation}
	f_{Y}{y}=\frac{y^{\nu-1}\alpha^{\nu}e^{-y\alpha}}{\Gamma(\nu)}
\end{equation}
we set the term A as the right side of the equation, and transforming the expression using $e$:$f_{Y}{y}=e^{ln(A)}$\\
and $ln(A)$ is:
\begin{equation}
	\frac{y\frac{-\alpha}{\nu}-(-log\alpha)}{\frac{1}{\nu}}+(\nu-1)logy-log\Gamma(\nu)
\end{equation}
setting two terms: $\theta = -\frac{\alpha}{\nu}$ and $\phi=\frac{1}{\nu}$
Therefore the density function can be expressed as:
\begin{equation}
	f_{Y}(y)=e^{\frac{y\theta-(-log(-\theta))}{\phi}-\frac{log\phi}{\phi}+(\frac{1}{\phi}-1)logy-log\Gamma(\frac{1}{\phi})}
\end{equation}
Therefore we have the distribution function of Gamma distribution is in exponential family with dispersion parameter $\phi \equiv \frac{1}{\nu}$, and variance of the distribution:
\begin{equation}
	Var(Y)=\frac{d^{2}ln(A)}{dy^{2}}=\frac{\nu}{(\alpha-y)^{2}}
\end{equation}
Which can be a value of dispersion parameter of Gamma distribution
\chapter{BPMN 2.0 notations}

\chapter{Raw results output}

\chapter{Code}

%%
%% NOTE that for this to typeset correctly, ensure you use the pdflatex
%%      command in preference to the latex command.  If you do not have
%%      the pdflatex command, you will need to remove the landscape and
%%      multicols tags and just make do with single column listing output
%%

\begin{landscape}
\begin{multicols}{2}
\section{File: yourCodeFile.java}
\lstinputlisting[basicstyle=\scriptsize]{yourCodeFile.java}
\end{multicols}
\end{landscape}

\end{document}

%%% Local Variables:
%%% mode: latex
%%% TeX-master: t
%%% End:
